% % \part{优化模型}
% % \chapter{约束非线性规划}

% \documentclass[UTF8]{ctexbook}

% \ctexset{
%     part/number = \chinese{part}
% }
% \usepackage{amsmath}% ams 数学公式
% \usepackage{mathtools}% ams 数学公式
% \usepackage{amsfonts}% ams 数学字体
% \usepackage{amssymb,latexsym}% ams 数学符号与LaTeX数学符号
% \usepackage{mathrsfs}% 花式符号
% \usepackage{ntheorem}%定理、定义、证明
%   \theoremstyle{nonumberplain}
%   \theoremheaderfont{\bfseries}
%   \theorembodyfont{\normalfont}
%   \theoremsymbol{$\square$}
%   \newtheorem{Proof}{\hskip 2em 证明}
%   \newtheorem{theorem}{\hspace{2em}定理}[chapter]
%   \newtheorem{definition}{\hspace{2em}定义}[chapter] % 如果没有章, 只有节, 把上面的[chapter]改成[section]
%   \newtheorem{axiom}[definition]{\hspace{2em}公理}
%   \newtheorem{lemma}[definition]{\hspace{2em}引理}
%   \newtheorem{proposition}[definition]{\hspace{2em}命题}
%   \newtheorem{corollary}[definition]{\hspace{2em}推论}
%   \newtheorem{remark}{\hspace{2em}注}[chapter] %类似地定义其他“题头”. 这里“注”的编号与定义、定理等是分开的

% \usepackage{enumerate}%itemiz环境。\begin{enumerate}[step 1]
% \usepackage{cite}%参考文献
%     \bibliographystyle{plain}
% \usepackage{extarrows}% 带参数的箭头
% \usepackage{hyperref}% 超链接
% %\usepackage[CJKbookmarks, colorlinks, bookmarksnumbered=true,pdfstartview=FitH,linkcolor=black,citecolor=black]{hyperref}%超链接的格式设置
% \hypersetup{
%     colorlinks=false,% 去掉超链接颜色
%     pdfborder=0 0 0% 取消超链接的边框
% }
% \usepackage{graphicx}% 图片管理
% \usepackage{caption}
% \usepackage{subcaption}%并排的图各有标题
% \graphicspath{{images/}}% 设置图片搜索路径
% \usepackage{float,varwidth}% 浮动体
% \usepackage{booktabs}% 三线表
% \usepackage{fancyhdr}% 页眉设置
% \usepackage{xcolor}% 颜色宏包
% \usepackage{colortbl}% 彩色表格
% \usepackage{listings}% 代码高亮
% \usepackage{caption}% 对标题进行控制,如让\caption标题的字体缩小一号,同时数字标签使用粗体可以用:\usepackage[font=small,labelfont=bf]{caption}
% \usepackage{xfrac,upgreek}%分别是行间公式如a/b的形式(将原来的命令\frac改成\sfrac)和希腊字体的宏包的
% \usepackage{mathtools}%lgathered和rgathered环境把公式向左向右对齐
% \usepackage{tabularx}%提供自动延伸的表列,(X列格式说明符),文字过长时可以自动转行
% \usepackage{longtable}%长表格
% \usepackage{enumitem}%enumerate宏包的升级
% \usepackage{harpoon}%数学公式的矢量
% \usepackage{bookmark}%目录的书签
% \usepackage{pifont}%给数字加上圈。然后在正文输入\ding{172}~\ding{211}得到相应数字,要是要①就输入:\ding{172}②就输:\ding{173}
% \renewcommand{\headwidth}{\textwidth}%图片并排,这个要列在所有宏包的后面
% \makeatletter
% \newcommand{\rmnum}[1]{\romannumeral #1}
% \newcommand{\Rmnum}[1]{\expandafter\@slowromancap\romannumeral #1@}
% \makeatother
% \definecolor{codegreen}{rgb}{0,0.6,0}
% \definecolor{codegray}{rgb}{0.5,0.5,0.5}
% \definecolor{codepurple}{rgb}{0.58,0,0.82}
% \definecolor{backcolour}{rgb}{0.95,0.95,0.92}
% \lstset{
%     commentstyle=\color{codegreen},
%     keywordstyle=\color{magenta},
%     numberstyle=\tiny\color{codegray},
%     stringstyle=\color{codepurple},
%     basicstyle=\footnotesize,
%     breakatwhitespace=false,% 断行只在空格处
%     breaklines=true,% 自动断行
%     captionpos=b,% 标题位置
%     keepspaces=true,
%     numbers=left,
%     numbersep=5pt,
%     showspaces=false,
%     showstringspaces=false,
%     showtabs=false,% 显示
%     tabsize=2% TAB 被当作两个空格
% }
% \topmargin=0pt\oddsidemargin=0pt\evensidemargin=0pt
% \textwidth=16.5cm\textheight=23cm\raggedbottom%我这么设置是为了缩小页边距,满足有的文字无法转行
% \pagestyle{headings}%页眉为章节标题,无页脚
% \setlength{\abovecaptionskip}{4pt}
% \setlength{\belowcaptionskip}{-8pt}%图片表格的前后距离设置
% \CTEXsetup[format={\zihao{-3}\raggedright\bfseries}]{section}%设置节的格式

% \begin{document}
% \part{优化模型}
\chapter{约束非线性规划}
\section{问题的引入与分析}
    \par
    前面介绍的是无约束非线性规划,如果我们在求最小化的过程中要求$x$满足一定的约束条件,则形成约束非线性规划问题。考虑如下含不等式约束的非线性优化问题
    \begin{align*}
    & \mathop {\min}\limits_x\  f(x)=x_1\exp(-(x_1^2+x_2^2))+(x_1^2+x_2^2)/20\\
    & s.t. \quad xy/2+(x+2)^2+(y-2)^2/2 \leqslant 2
    \end{align*}
    其中:$x=(x_1,x_2) \in R^2$。
    \par
    Optimization Toolbox使用四种算法来求解约束非线性规划问题:
    \ding{172}内点算法:特别适用于具有稀疏性或其他结构的大规模问题,它基于障碍函数,且在优化迭代过程中对于边界严格可行;
    \ding{173}SQP算法;
    \ding{174}动态序列算法;
    \ding{175}信赖域反射算法:仅用于边界约束或线性等。在内点算法和信赖域反射算法中,对Hesse矩阵的近似:\\
    (1)对内点法而言,可以通过以下方法来近似Hesse矩阵:\par
    1)\ BFGS(稠密);\par
    2)有限内存BFGS(用于大规模问题);\par
    3)海赛 - 乘函数;\par
    4)实际海赛矩阵(稀疏式稠密);\par
    5)有限差分法,但不要求预先知道稀疏性结构。\\
    (2)对信赖域反射算法,可以通过以下方法来近似Hesse矩阵:\par
    1)有限差分法;\par
    2)实际海赛矩阵(稀疏式稠密);\par
    3)海赛 - 乘函数;\par
    用MATLAB求解上面的问题,程序如下
    \begin{lstlisting}[language=Matlab]
    f = @(x,y) x.*exp(-x.^2-y.^2)+(x.^2+y.^2)/20;
    g = @(x,y) x.*y/2+(x+2).^2+(y-2).^2/2-2;
    ezplot(g,[-6,0,-1,7])
    hold on
    ezcontour(f,[-6,0,-1,7])
    plot(-.9727,.4685,'ro');
    legend('constraint','f contours','minimum');
    hold off
    x0 = [-2 1];
    options = optimoptions('fmincon','Algorithm','interior-point','Display','iter');% 求解器fmincon使用内点算法(interior-point aligorithm),并打开每一次迭代的结果。
    gfun = @(x) deal(g(x(1),x(2)),[]);% 求解要求非线性约束有两个:一个是非线性不等式,另一个是非线性等式。我们用deal函数写这个约束。
    [x,fval,exitflag,output] = fmincon(fun,x0,[],[],[],[],[],[],gfun,options);
    \end{lstlisting}
\section{模型规范化及基本理论}
    \par
    我们将前面的引例规范化,写出约束非线性规划的一般形式
    \begin{align*}
    &\mathop {\min}\limits_{x\in R^n} \ f(x)\\
    &s.t.\left\{
    \begin{aligned}
    &h_i(x)=0\quad i=1,2,\ldots,l\\
    &g_j(x) \geqslant 0\quad j=1,2,\ldots,m
    \end{aligned}
    \right.
    \end{align*}
    其中:$h_i(x),g_j(x)$为定义在$R^n$上的实值连续函数,$h_i(x)=0$是等式约束,$g_j(x) \geqslant 0$是不等式约束,$f$是目标函数。如果$m=0$,即不存在不等式约束,则称规划问题为非线性等式约束规划;如果$h_i(x),g_j(x)$为线性函数,则称为线性约束规划。进一步,如果一个线性约束规划的目标函数$f$为二次函数,则称为二次规划。二次规划是最简单的约束非线性规划问题。
    \par
    \ding{172}量的规定:
    $x\in R^n$;
    $f(x):R^n \to R$;
    $h_i(x):R^n \to R$;
    $g_j(x):R^n \to R$;
    至于$f,h,g$的具体函数类型则有待进一步讨论,例:$f \in C^1(D)$或者$f \in C^{2}(D)$。
    \par
    \ding{173}解空间的定义(可行域):满足约束$h_i(x)=0$和$g_j(x) \geqslant 0$的解$x$称为可行解,可行解集合称为可行域,记为$D=\{x|h_i(x)=0 \ \&\  g_j(x) \geqslant 0\}$。我们的目标是在$D$中求$x^*$使得$f$最小。
    \par
    \ding{174}解的定义(全局极小点,局部极小点):\par
    % 使$f$最小的解称为最优解,记为$x^*$。如果$x^*$是在$D$中取值,则称之为全局极小点;如果$x^*$仅在$D$的局部取值,则称为局部极小点。
    \begin{definition}[全局极小点]
    设$x^* \in D$,如果对$\forall x \in D$,有
    \begin{align*}
    f(x) \geqslant f(x^*)
    \end{align*}
    则称$x^*$为全局极小点。如果对$\forall x \in D,x \neq x^*$,有
    \begin{align*}
    f(x) > f(x^*)
    \end{align*}
    则称$x^*$为全局严格极小点。
    \end{definition}
    \begin{definition}[局部极小点]
    设$x^* \in D$,如果$\exists \delta >0,\forall x \in D \cap N(x^*,\delta)$,有
    \begin{align*}
    f(x) \geqslant f(x^*)
    \end{align*}
    则称$x^*$为$x$的$\delta$局部极小点\footnote{注:在全局最优章节之前,我们讨论的$x^*$皆为局部极小点。}。其中:$N(x^*,\delta)=\{x\big|\|x-x^*\|_2 \leqslant \delta\}$。
    如果对$\forall x \in D \cap N(x^*,\delta)/x^*$,有
    \begin{align*}
    f(x) > f(x^*)
    \end{align*}
    则称$x^*$为严格局部极小点。
    \end{definition}
    \par
    毫无疑问,全局极小值$f(x^*) \leqslant \text{局部极小值}f(x^*)$,某种特殊情况下等式成立,这是有待讨论的。
\section{解的存在性 - 最优性条件}
    \par
    假设$f,h_i,g_j$都是连续可微的。仅依据局部极小点的定义来判断一个$x$是否为$x^*$是困难的,因此,我们有必要给出极小点存在的充分必要条件,以方便计算极小点。
    \par
    我们先给出只有等式约束$h_i(x)=0$时解的最优性条件,然后再讨论不等式约束$g_j(x) \geqslant 0$,最终将二者结合。这样做的理由是:等式约束是相对易于处理的,并且在数学分析中,我们也学过条件极值和拉格朗日乘数法。
    \subsection{等式约束的最优化条件}
        \par
        考虑如下等式约束规划问题
        \begin{align}
        \label{eq等式约束规划问题}
        &\mathop {\min}\limits_{x\in R^n}\  f(x)\\
        &s.t.\quad h_i(x)=0\quad i=1,2,\ldots,l\notag
        \end{align}
        我们用拉格朗日函数处理上述等式约束极值问题,做Largrange函数
        \begin{align*}
        L(x,\lambda)=f(x)-\mathop {\sum}\limits_{i=1}^l{\lambda}_i h_i(x)=\lambda^\mathrm{T}h(x)
        \end{align*}
        其中:$\lambda=({\lambda}_1,{\lambda}_2,\ldots,{\lambda}_l)^\mathrm{T} $为拉格朗日乘子。
        \paragraph{一阶必要条件}
        设$x^*$是问题(\ref{eq等式约束规划问题})的局部极小点,$f,h_i$在$x^*$的某邻域内连续可微,若向量组$\nabla h_i(x^*)$线性无关,则$\exists {\lambda}^*=({\lambda}_1^*,{\lambda}_2^*,\ldots,{\lambda}_l^*)^\mathrm{T} $使得
        \begin{align*}
        {\nabla}_xL(x^*,{\lambda}^*)=0
        \end{align*}
        即
        \begin{align*}
        \nabla f(x^*) - \sum_{i=1}^l\lambda_i^*\nabla h_i(x^*) = 0
        \end{align*}
        \paragraph{二阶充分条件}
        定义$L(x,\lambda)$的梯度以及关于$x$的Hesse矩阵
        \begin{align*}
        \nabla L(x,\lambda)=\bigg(\begin{aligned}{\nabla}_x L(x,\lambda)\\
        {\nabla}_{\lambda} L(x,\lambda)
        \end{aligned}\bigg)
        =
        \begin{pmatrix}
        \nabla f(x)-\mathop {\sum}\limits_{i=1}^l{\lambda}_i \nabla h_i(x)\\
        -h(x)
        \end{pmatrix}
        \end{align*}
        \begin{align*}
        {\nabla}_{xx}^2L(x,\lambda)={\nabla}^2f(x)-\mathop {\sum}\limits_{i=1}^l{\lambda}_i {\nabla}^2 h_i(x)\triangleq G
        \end{align*}
        设$f,h_i$是二阶连续可微,并$\exists(x^*,{\lambda}^*) \in R^n\times R^l$使$\nabla L(x^*,{\lambda}^*)=0$。若$\forall d \in R^n/0$,$\nabla h_i(x^*)^\mathrm{T} d=0$,有$d^\mathrm{T} {\nabla}_{xx}^2L(x^*,\lambda)d>0$,则$x^*$是优化问题的一个严格局部极小点。
    \subsection{不等式约束问题的最优性条件}
        \par
        考虑如下不等式约束优化问题
        \begin{align*}
        & \min_{x\in R^n}\ f(x)\\
        & s.t.\quad g_j(x) \geqslant 0\quad j=1,2,\ldots,m
        \end{align*}
        记可行域为$D=\{x|g_j(x) \geqslant 0\}$,指标集$I=\{1,2,\ldots,m\}$。对于某一个可行点$x \in D$而言,其s.t.会有两种情况:有些约束是$g_j(x)=0$,另一些约束是$g_j(x)>0$。对于后一种情形,在$x$的某个邻域内仍然保持$g_j(x)>0$成立,而前者不具备这种性质。因此,我们将二者分开,定义积极集:
        \begin{definition}[积极集]
        若某个可行点$x \in D$,使得$g_j(x)=0,j\in I$,则称不等式约束$g_j(x) \geqslant 0$为$x$的有效约束,称集合$I(x)=\{j:g_j(x)=0\}$为$x$处的有效约束指标集。
        \end{definition}
        \par
        给出上述问题的广义拉格朗日函数
        \begin{align*}
        L(x,\mu)=f(x)-{\mu}^\mathrm{T} g(x)=f(x)-\mathop {\sum}\limits_{j=1}^m \mu_j g_j(x)
        \end{align*}
        其中:$\mu= (\mu_1,\dots,\mu_m)$为广义拉格朗日乘子。
        \paragraph{一阶必要条件(KKT条件或KT条件)}
        设$x^*$为极小点,有效约束指标集$I(x^*)=\{j|g_j(x^*)=0,j \in I\}$,假设$f,g_j$在$x^*$处可微,若向量组${\nabla}{g_j}(x^*),j \in I(x^*)$线性无关,则$\exists {\mu}^*=({\mu}_1^*,{\mu}_2^*,\ldots,{mu}_m^*)^\mathrm{T} $使得
        \begin{align*}
        \left \{
        \begin{aligned}
        & \nabla f(x^*)-\mathop {\sum}\limits_{j=1}^m {\mu}^*_i{\nabla}{g_j}(x^*)=0\\
        & g_j(x^*) \geqslant 0\\
        & {\mu}^*_j \geqslant 0\\
        & {\mu}^*_j g_j(x^*)=0
        \end{aligned}
        \right.
        \end{align*}
        \par
        反过来,如果$\exists x^* \in D,{\mu}^* \in R^m$使得上述KKT条件成立,则$x^*$为最优化问题的K-T点。由上述必要性易知,极小点$\Rightarrow$K-T点,但K-T点$\nRightarrow$极小点。
        \par
        由于KKT条件是1951年Kuhn和Tucher给出,故它常被称为K-T条件。同时,1939年Karush也类似地考虑了约束的最优性条件,所以也被称为K-K-T条件(Karush-Kuhn-Tucher定理)。
    \subsection{一般约束问题的最优性条件}
        \par
        上面,我们将等式约束和不等式约束分开讨论,下面来讨论如下一般约束最优问题
        \begin{align*}
        &\mathop {\min}\limits_{x\in R^n} \ f(x)\\
        &s.t.\left\{
        \begin{aligned}
        &h_i(x)=0\quad i=1,2,\ldots,l\\
        &g_j(x) \geqslant 0\quad j=1,2,\ldots,m
        \end{aligned}
        \right.
        \end{align*}
        记指标集$E=\{1,2,\ldots,l\},I=\{1,2,\ldots,m\},i \in E,j \in I$,可行域为$D=\{x \in R^n|h_i(x)=0,g_j(x) \geqslant 0,\forall i \in E,j \in I\}$。
        \par
        定义上述优化问题的广义拉格朗日函数为
        \begin{align*}
        L(x,\lambda,\mu)=f(x)-\mathop {\sum}\limits_{i=1}^l {\lambda}_ih_i(x)-\mathop {\sum}\limits_{j=1}^m {\mu}_jg_j(x)
        \end{align*}
        其中:$\mu ,\lambda$为广义拉格朗日乘子。
        \paragraph{一阶必要条件(KKT条件)}
        设$x^*$是局部极小点,在$x^*$处的有效约束集为
        \begin{align*}
        s(x^*)=E\cup I(x^*)=E\cup\{j|g_j(x^*)=0,i\in I\}
        \end{align*}
        并假设$f,h_i,g_j$在$x$处可微。若向量组$\nabla h_i(x^*),\nabla g_j(x^*),j \in I(x^*)$线性无关,则存在向量$({\lambda}^*,{\mu}^*) \in R^l\times R^m$,得到
        \begin{align*}
        \left \{
        \begin{aligned}
        & \nabla f(x^*)-\mathop {\sum}\limits_{i=1}^l {\lambda}^*_i{\nabla}{h_i}(x^*)-\mathop {\sum}\limits_{j=1}^m {\mu}^*_j{\nabla}{g_j}(x^*)=0\\
        & h_i(x^*)=0,i \in E\\
        & g_j(x^*) \geqslant 0\\
        & {\mu}^*_j \geqslant 0\\
        & {\mu}^*_j g_j(x^*)=0,j \in I
        \end{aligned}
        \right.
        \end{align*}
        其中:$\mu ,\lambda$为广义拉格朗日乘子。称${\mu}_j^*g_j(x^*)=0(j \in I(x^*))$为互补松弛条件。这意味着${\mu}_j^*,g_j(x^*)$中至少有一个必为0。
        \par
        定义广义拉格朗日函数$L$关于$x$的梯度和Hesse矩阵为
        \begin{align*}
        & {\nabla}_{x}L(x,\lambda,\mu)={\nabla}f(x)-\mathop {\sum}\limits_{i=1}^l{\lambda}_i {\nabla} h_i(x)-\mathop {\sum}\limits_{j=1}^m{\mu}_j {\nabla} g_j(x)\\
        & {\nabla}_{xx}L(x,\lambda,\mu)={\nabla}^2f(x)-\mathop {\sum}\limits_{i=1}^l{\lambda}_i {\nabla}^2 h_i(x)-\mathop {\sum}\limits_{j=1}^m{\mu}_j {\nabla} g_j^2(x)
        \end{align*}
        \paragraph{二阶充分条件}
        假设$f,g,h$是二阶连续可微的,设$(x^*,({\mu}^*,{\lambda}^*))$是优化问题的一个KKT点(即$(x^*,({\mu}^*,{\lambda}^*))$满足KKT条件),若$\forall d \in R^n/0$,$\nabla g_j(x^*)^\mathrm{T} d=0(j \in I(x^*))$,$\nabla h_i(x^*)^\mathrm{T} d=0(i \in E)$均有$d^\mathrm{T} {\nabla}_{xx}L(x^*,{\lambda}^*,{\mu}^*)d>0$,则$x^*$是一个严格局部极小点。
        \par
        由一阶必要条件,我们知道,优化问题的KKT点不一定是局部极小点,但如果优化问题是一个凸优化问题,则KKT点$\equiv$局部极小点$\equiv$全局极小点。
        凸优化是如下优化问题
        \begin{align*}
        &\mathop {\min}\limits_{x\in R^n} f(x)\\
        &s.t.\left\{
        \begin{aligned}
        &h_i(x)=0\quad i=1,2,\ldots,l\\
        &g_j(x) \geqslant 0\quad j=1,2,\ldots,m
        \end{aligned}
        \right.
        \end{align*}
        如果$f$是凸函数,$h_i(x)$是线性函数(仿射函数),$g_j(x)$是凹函数(即$-g_j(x)$是凸函数),那么该优化问题为凸优化。\\
\section{对偶问题与鞍点}
    \par
    考虑如下约束非线性规划问题
    \begin{align*}
    &\mathop {\min}\limits_{x\in R^n}\  f(x)\\
    &s.t.\left\{
    \begin{aligned}
    &h_i(x)=0\quad i \in E\\
    &g_j(x) \geqslant 0\quad j \in I
    \end{aligned}
    \right.
    \end{align*}
    称上述问题为此非线性规划的原始问题(PNLP),相对的对偶问题(DNLP)定义如下
    \begin{align*}
    & \max\  \theta(\lambda,\mu)=\inf L(x,\lambda,\mu)=\inf\{f(x)-\lambda^\mathrm{T}h(x) -\mu^\mathrm{T}g(x) |x \in D\}\\
    & s.t.\quad \mu \geqslant 0
    \end{align*}
    其中:$\theta(\lambda,\mu)$称为原问题的拉格朗日对偶函数。此外,如果设原问题和对偶问题的可行域分别为$D$和$\Delta$,相应的拉格朗日函数为
    \begin{align*}
    L(x,\lambda,\mu)=f(x)-\lambda^\mathrm{T}h(x) -\mu^\mathrm{T}g(x)  ,\quad x \in D,\mu \in R_+^m,\lambda \in R^l
    \end{align*}
    亦可记$m=|I|,l=|E|$。值得指出的是,对$\forall x \in D$,拉格朗日函数$L(x,\lambda,\mu)$是$\lambda,\mu$的线性函数,于是,拉格朗日对偶函数$\theta(\lambda,\mu)$作为线性函数的逐点下确界,必然是一个凹函数。这说明上述对偶问题是一个凸规划问题。
    \par
    上面,我们给出了对偶问题的定义,需要明确的是:原始问题的最优值是否等于对偶问题的最优值呢?即$\mathop {\min}\limits_{x \in s}f(x)=\mathop {\max}\limits_{(\lambda,\mu) \in \Delta }\theta(\lambda,\mu)$?
    \subsection{弱对偶定理}
        \par
        设$x \in D,(\lambda,\mu) \in \Delta$分别为原问题和对偶问题的可行解,则
        \begin{align*}
        f(x) \geqslant \theta(\lambda,\mu)
        \end{align*}
        对于原问题(PNLP)和对偶问题(DNLP),下面3个结论成立:\par
        1)若$D \neq \phi,\Delta \neq \phi$,则
        \begin{align*}
        f_{min} = \inf\{f(x)|x \in D\} \geqslant {\theta}_{max}=\sup\{\theta(\lambda,\mu)|(\lambda,\mu) \in \Delta\}
        \end{align*}
        \par
        2)若$\exists x^* \in D$和$({\lambda}^*,{\mu}^*)\in \Delta$,使得$f(x^*)\leqslant \theta({\lambda}^*,{\mu}^*)$,则$x^*$和$({\lambda}^*,{\mu}^*)$的解为原问题和对偶问题的最优解。
        \par
        3)若$f_{min}=-\infty$,则$\theta(\lambda,\mu) \in \Delta,\theta(\lambda,\mu)=-\infty$。若${\theta}_{max}=+\infty$,则原问题没有可行解。
        \par
        对于一般问题,我们知道$\delta=f_{min}-{\theta}_{max} \neq 0$,下面,我们讨论在什么条件下$\delta = 0$?

    \subsection{强对偶定理}
        \begin{definition}[slater约束规格]
        对于凸优化来说,若相对内点的集合$riS=\{x|h(x)=0,g(x)>0,x\in D\}\neq \phi$,则称约束函数满足slater约束规格。
        \end{definition}
        \par
        对于凸优化而言,假设$D$为非空的凸开集,$f$为凸函数,$g_j(j\in I)$为凹函数,$h_i(i\in E)$为线性函数。若函数$g,h$满足slater约束规格,则
        \begin{align*}
        f_{min} := \inf\{f(x)|x \in D\} =\sup\{\theta(\lambda,\mu)|(\lambda,\mu) \in \Delta\}=:{\theta}_{max}
        \end{align*}
        此外,若$f_{min}>-\infty$,则$\exists ({\lambda}^*,{\mu}^*)\in \Delta$,使得$\theta({\lambda}^*,{\mu}^*)={\theta}_{max}$。特别地,若存在$x^* \in s$,使$f(x^*)=f_{min}$,则互补松弛条件${\lambda}^{*T}g(x^*)=0$成立。
    \subsection{鞍点定理}
        \par
        下面介绍鞍点的定义以及鞍点与KKT点、最优解之间的关系。
        \begin{definition}[鞍点]
        \par
        设$L(x,\lambda,\mu)$是原问题的拉格朗日函数,如果$\exists (x^*,{\lambda}^*,{\mu}^*)\in D\times R^{|E|}\times R_+^{|I|}$,使得对$\forall(x,\lambda,\mu)$,有
        \begin{align*}
        L(x^*,{\lambda},{\mu}) \leqslant L(x^*,{\lambda}^*,{\mu}^*) \leqslant L(x,{\lambda}^*,{\mu}^*)
        \end{align*}
        则称$(x^*,{\lambda}^*,{\mu}^*)$为函数$L(x,\lambda,\mu)$的一个鞍点(saddle point)。
        \end{definition}
        \subsubsection{鞍点与最优点}
            \par
            在一般情况下,鞍点的存在性并不是最优解存在的必要条件,但是,如果鞍点存在的话,它一定是约束优化问题原问题的最优解。
            \par
            1)设$(x^*,{\lambda}^*,{\mu}^*)$是原问题的拉格朗日函数$L(x,\lambda,\mu)$的鞍点,则$x^*$和$({\lambda}^*,{\mu}^*)$分别是原问题和对偶问题的最优解。
            \par
            2)如果原问题为凸优化问题,$x^*$为最优解,并且$g_j(x),h_i(x)$满足slater约束规格,则$\exists {\lambda}^* ,{\mu}^*\geqslant 0$,使得$(x^*,{\lambda}^*,{\mu}^*)$是原问题的拉格朗日函数的鞍点。
        \subsubsection{鞍点与KKT点}
            \par
            假设原问题(PNLP)中的$f,g,h$在包含可行域$D$的开集$D$内连续可微,那么:
            \par
            1)若$(x^*,{\lambda}^*,{\mu}^*)\in D\times R_+^{|I|}\times R^{|E|}$是原问题的拉格朗日函数$L(x,\lambda,\mu)$的鞍点,则$(x^*,{\lambda}^*,{\mu}^*)$是原问题(PNLP)的一个KKT点对。
            \par
            2)若原问题是一个凸规划,并且$(x^*,{\lambda}^*,{\mu}^*)$是原问题的一个KKT点,则$(x^*,{\lambda}^*,{\mu}^*)$是原问题的拉格朗日函数的鞍点。
\section{最优化算法}
    \subsection{外罚函数法}
        \par
        罚函数的基本思想:根据约束条件的特点,将其转化为某种罚函数加到目标函数中去,从而将约束优化问题转化为无约束优化问题来求解。下面,我们来介绍一种罚函数方法 - 外罚函数法,也称外点法。
        \par
        考虑如下一般的约束优化问题
        \begin{align*}
        &\mathop {\min}\limits_{x\in R^n}\  f(x)\\
        &s.t.\left\{
        \begin{aligned}
        &h_i(x)=0\quad i \in E\\
        &g_j(x) \geqslant 0\quad j \in I
        \end{aligned}
        \right.
        \end{align*}
        记可行域为$D=\{x|h_i(x)=0,g_j(x) \geqslant 0\},|E|=l,|I|=m$,构造罚函数
        \begin{align*}
        \bar{p}(x)=\mathop {\sum}\limits_{i=1}^lh_i^2(x) + \mathop {\sum}\limits_{j=1}^m[min\{0,g_j(x)\}]^2
        \end{align*}
        将罚函数$\bar{p}(x)$增加到目标中,构建新的目标函数
        \begin{align*}
        p(x,\sigma)=f(x)+\sigma \bar{p}(x)
        \end{align*}
        其中:$\sigma$为罚权重,$\sigma >0$。
        不难发现,当$x \in D$时,$\min f(x) \triangleq \min p(x,\sigma)$;当$x \notin D$时,$p(x,\sigma)>f(x)$,$\sigma$越大,$p$越大,当$\sigma$足够大时,要使$p$达到极小,罚函数$\bar{p}(x)$要充分小才可以,从而$p(x,\sigma)$的极小点充分逼近可行域$D$,因此,最优化问题等价于
        \begin{align*}
        \mathop {\min}\limits_{x \in R^n}p(x|{\sigma}_k)=f(x)+{\sigma}_k\bar{p}(x)
        \end{align*}
        其中:${\sigma}_k$为第$k$次迭代的罚权重,${\sigma}_k>0$,且${\sigma}_k \to +\infty$。
        \par
        由上述思想可知:$x(\sigma)$是从可行域$D$的外部来趋近于$x^*$,因此上述罚函数也称为外罚函数/外点法。(问:如何确定初点$x_0 \notin D$)。上述思路有一些缺点:\ding{172}当${\sigma}_k$较大时,$p(x,{\sigma}_k)$的Hesse矩阵的条件数很大,在数值上求解不易。\ding{173}$\bar{p}(x)$一般不可微,因此,不易于直接采用导数求解方法。
    \subsection{内罚函数法 - 内点法}
        \par
        (1)对于只有不等式约束的优化问题
        \begin{align*}
        &\mathop {\min}\limits_{x\in R^n}\  f(x)\\
        &s.t.\quad g_j(x) \geqslant 0\quad j \in I
        \end{align*}
        内点法的基本思想是:保持每一个迭代点$x_k$都在可行域$D$内,可行域的边界被筑起一道很高的“围墙”作为障碍。当迭代点$x_k$靠近边界时,增广目标函数值骤然增大,以示“惩罚”,并阻止迭代点窜越边界。因此,内点法也称为内罚函数法或障碍函数法。它只适用于可行域内点集非空的情形。
        \par
        类似于外罚函数法,我们构造增广目标函数
        \begin{align*}
        H(x,\tau)=f(x)+\tau \bar{H}(x)
        \end{align*}
        其中;$\bar{H}(x)$为障碍函数。当$x \in D$,至少有一个$g_j(x)$趋近于0,而$\bar{H}(x)$趋近于无穷大。因此,可取约束函数倒数之和为障碍函数
        \begin{align*}
        \bar{H}(x)=\mathop {\sum}\limits_{j=1}^m \frac{1}{g_j(x)}
        \end{align*}
        或者反对数障碍函数
        \begin{align*}
        \bar{H}(x)=-\mathop {\sum}\limits_{j=1}^m \ln(g_j(x))
        \end{align*}
        $\tau>0$为罚权重,$\tau_k \to 0(k \to \infty)$。于是,约束优化问题转化为无约束优化问题
        \begin{align*}
        \mathop {\min}\limits_{x}\ H(x,\tau)=f(x)+\tau \bar{H}(x)
        \end{align*}
        问:如何确定初始点$x \in D$?
        \par
        (2)如果约束中存在等式约束,我们可以将外罚函数法和内罚函数法相结合,构建增广目标函数
        \begin{align*}
        H(x,\mu)=f(x)+\frac{1}{2\mu}\mathop {\sum}\limits_{i=1}^l h_i^2(x)+\mu\mathop {\sum}\limits_{j=1}^m \frac{1}{g_j(x)}
        \end{align*}
        或者
        \begin{align*}
        H(x,\mu)=f(x)+\frac{1}{2\mu}\mathop {\sum}\limits_{i=1}^l h_i^2(x)-\mu\mathop {\sum}\limits_{j=1}^m \ln[{g_j(x)}]
        \end{align*}
        另外,我们还可以引入松弛变量${\xi}_j,j=1,2,\ldots,m$,将原问题转化为
        \begin{align*}
        &\mathop {\min}\  f(x)\\
        &s.t.\left\{
        \begin{aligned}
        &h_i(x)=0\quad i=1,2,\ldots,l\\
        &g_j(x)-{\xi}_j = 0\quad j=1,2,\ldots,m\\
        &{\xi}_j\quad j=1,2,\ldots,m
        \end{aligned}
        \right.
        \end{align*}
        然后构建混合增广目标函数
        \begin{align*}
        \varphi(x,\xi,\mu)=f(x)+\frac{1}{2\mu}\mathop {\sum}\limits_{i=1}^l h_i^2(x)+\frac{1}{2\mu}\mathop {\sum}\limits_{j=1}^m [{g_j(x)}-{\xi}_j]^2+\mu\mathop {\sum}\limits_{j=1}^m \frac {1}{{\xi}_j}
        \end{align*}
    \subsection{乘子法}
        \par
        乘子法是Povell和Hestenes于1969年针对等式约束问题同时独立提出的。Rockfellar于1973年将该方法推广到不等式约束优化中。其基本思想是:从原问题的拉格朗日函数出发,再加上适应的罚函数,从而将原问题转化为一个无约束优化问题。
        \par
        先只考虑等式约束优化问题
        \begin{align*}
        &\mathop {\min}\  f(x)\\
        &s.t.\quad h_i(x) = 0\quad i \in E
        \end{align*}
        作上述问题的拉格朗日函数
        \begin{align*}
        L(x,\lambda)=f(x)-{\lambda}^\mathrm{T}  h(x)
        \end{align*}
        其中:${\lambda}^\mathrm{T} =({\lambda}_1,{\lambda}_2,\ldots,{\lambda}_l)^\mathrm{T} $为Lagrange乘子向量。$h(x)=(h_1(x),h_2(x),\ldots,h_l(x))^\mathrm{T} $。
        \par
        设$(x^*,{\lambda}^*)$是原问题的KKT点,则由最优性条件,有
        \begin{align*}
        & {\nabla}_xL(x^*,{\lambda}^*)=0\\
        & {\nabla}_{\lambda}L(x^*,{\lambda}^*)=-h(x^*)=0
        \end{align*}
        此外,对$\forall x \in D$,有
        \begin{align*}
        L(x^*,{\lambda}^*)=f(x^*) \leqslant f(x)-({\lambda}^*)^\mathrm{T} h(x)=L(x,{\lambda}^*)
        \end{align*}
        上式表明,如果已知${\lambda}^*$,则原问题等价于
        \begin{align*}
        &\mathop {\min}\ L(x,{\lambda}^*)\\
        &s.t.\quad h(x) = 0
        \end{align*}
        可以考虑用外罚函数法求解上述问题,其增广目标函数为
        \begin{align*}
        \varphi(x,{\lambda}^*,\sigma)=L(x,{\lambda}^*)+\frac{\sigma}{2}\|h(x)\|^2
        \end{align*}
        但${\lambda}^*$事先并不知道,故可以考虑如下增广目标函数
        \begin{align*}
        \varphi(x,{\lambda},\sigma)&=L(x,{\lambda})+\frac{\sigma}{2}\|h(x)\|^2\\
        &=f(x)-{\lambda}^\mathrm{T} h(x)+\frac{\sigma}{2}\|h(x)\|^2
        \end{align*}
        我们求$\mathop {\min}\ \varphi(x,{\lambda},\sigma)$。首先固定一个$\lambda=\bar{\lambda}$,求$\varphi(x,\bar{\lambda},\sigma)$的极小点$\bar{x}$;然后再适当改变$\lambda$的取值,再求新的$\bar{x}$,直到求得满意的$x^*,{\lambda}^*$为止。
        \par
        具体而言,我们在第$k$次迭代求无约束的问题$\mathop {\min}\ \varphi(x,{\lambda}_k,\sigma)$的极小点$x_k$时,其取极值的必要条件为
        \begin{align*}
        {\nabla}_x\varphi(x_k,{\lambda}_k,{\sigma})={\nabla}f(x_k)-{\nabla}h(x_k)[{\lambda}_k-{\sigma}h(x_k)]=0
        \end{align*}
        而在原问题的KKT点$(x^*,{\lambda}^*)$处,有
        \begin{align*}
        {\nabla}f(x^*)-\nabla h(x^*){\lambda}^*=0\quad h(x^*)=0
        \end{align*}
        我们自然是希望$\{x_k\} \to x^*,\{{\lambda}_k\} \to {\lambda}^*$,于是,可以取${\lambda}^*$的更新公式为
        \begin{align*}
        {\lambda}_{k+1}={\lambda}_k-{\sigma}h(x_k)
        \end{align*}
        且$\{h(x_k)\} \to 0$是$\{{\lambda}_k\}$收敛的充要条件,也是$(x_k,{\lambda}_k)$为KKT对的充要条件。
        \par
        上面讨论的是只含有等式约束的优化问题,如果含有不等式约束,我们可以引入松弛变量${\xi}_j$以进行转化。一般约束最优化的增广Lagrange函数为
        \begin{align*}
        L(x,\lambda,\mu)&=f(x)+\frac{1}{2\mu}\mathop {\sum}\limits_{j \in I}\left( \mathop{\min}{}^2\{\mu g_j(x)-{\lambda}_j,0\}-{\lambda}_j^2 \right)  \\
        &\quad -\mathop {\sum}\limits_{i \in E}{\lambda}_ih_i(x)+ \frac {1}{2}\mu\mathop {\sum}\limits_{i \in E}h_i^2(x)
        \end{align*}
        相应的Lagrange乘子迭代格式为
        \begin{align*}
        {\lambda}_i^{+} = \left\{
        \begin{aligned}
        &{\lambda}_i-\mu h_i(x)\quad i \in E\\
        &\mathop {\max}\{{\lambda}_j-\mu g_j(x),0\}\quad j\in I
        \end{aligned}
        \right.
        \end{align*}
        其中:$x,\mu,\lambda$为当前迭代值,${\lambda}_i^{+}$表示下一次迭代值。
        \par
        给出一般的乘子法算法流程:\\
        \textbf{step1.}初始化。
        $x_0 \in R^n$,初始乘子向量${\lambda}_0$,罚参数序列$\{{\mu}_k\}$,容许误差$\varepsilon > 0,k:=0$。\\
        \textbf{step2.}构建增广Lagrange函数$L(x,\lambda,\mu)$。\\
        \textbf{step3.}以$x_{k-1}$作为初始点($k=0$时,初始点任意),求无约束优化
        \begin{align*}
         \mathop {\min}\limits_{x \in R^n}\ L(x,{\mu}_k,{\lambda}_k)
        \end{align*}
        解得$x_k$。\\
        \textbf{step4.}若
        \begin{align*}
         \|h(x_k)\|+\|{\min}\{g(x_k),{\mu}_k^{-1}{\lambda}_k\}\| \leqslant \varepsilon
        \end{align*}
        则解得$x_k$,否则转到step5。\\
        \textbf{step5.}更新${\lambda}_k$。令$x:=x_k,\lambda:={\lambda}_k$,由${\lambda}_i^+ $更新公式得到${\lambda}_{k+1}$,转至step2。
    \subsection{SQP方法(序列二次规划方法)}
        \par
        SQP(sequential quadratic programing)是求解约束优化问题最有效的算法之一。其基本思想是:在每一次迭代步通过求解一个二次规划子问题来确定一个下降方向。然后,以减少价值函数来取得步长,重复这些步骤直到收敛。
        \subsubsection{Newton - Lagrange方法}
            \par
            先来考虑等式约束
            \begin{align*}
            &\mathop {\min}\limits_{x\in R^n} \ f(x)\\
            &s.t.\quad h(x) = 0
            \end{align*}
            其中:$f:R^n \to R,h_i:R^n \to R,h(x)=(h_1(x),h_2(x),\ldots,h_l(x))^\mathrm{T} ,E=\{1,2,\ldots,l\},|E|=l$。设$f,h_i$为二阶连续可微函数。
            \par
            写出原问题的Lagrange函数
            \begin{align*}
            L(x,\mu)=f(x)-{\mu}^\mathrm{T}  h(x)
            \end{align*}
            记约束$h$的梯度矩阵为
            \begin{align*}
            \nabla h(x)=(\nabla h_1(x),\nabla h_2(x),\ldots,\nabla h_l(x))
            \end{align*}
            $h$的Jacobi矩阵为$A(x)=\nabla h(x)^\mathrm{T} $。根据原问题的KKT条件,可得到如下方程组
            \begin{align}
            \label{eqSQP方法1}
            \nabla L(x,\mu)=\bigg(\begin{aligned}{\nabla}_x L(x,\mu)\\
            {\nabla}_{\mu} L(x,\mu)\end{aligned}\bigg)
            =
            \begin{pmatrix}
            \nabla f(x)-A(x)^\mathrm{T} {\mu}\\
            -h(x)
            \end{pmatrix}
            =0
            \end{align}
            若$A(x^*)$行满秩,则原问题的最优解$(x^*,{\mu}^*)$必然满足上述非线性方程(\ref{eqSQP方法1})。由于KKT条件为Lagrange函数平稳点的条件,所以人们通常把基于求解上述非线性方程的优化方法称为Lagrange方法。特别地,如果使用Newton方法求解上述方程组,那么相应的优化方法称为Newton-Lagrange方法。
            \par
            现在用牛顿法求解上述的非线性方程组(\ref{eqSQP方法1})。记函数$\nabla L(x,\mu)$的Jacobi矩阵为
            \begin{align*}
            N(x,\mu)=\bigg[\begin{matrix}& G(x,\mu)\quad & -A(x)^\mathrm{T} \\
            & -A(x)\quad & 0\end{matrix}\bigg]
            \end{align*}
            其中:
            \begin{align*}
            G(x,\mu)={\nabla}_{xx}^2L(x,\mu)={\nabla}^2f(x)-\mathop {\sum}\limits_{i=1}^l{\mu}_i {\nabla}^2 h_i(x)
            \end{align*}
            为Lagrange函数$L(x,\mu)$关于$x$的Hesse矩阵。
            \par
            对于给定的点$(x^*,{\mu}^*)$,牛顿法的迭代公式为
            \begin{align*}
            \bigg(\begin{aligned}x^{k+1}\\{\mu}^{k+1}\end{aligned}\bigg)=\bigg(\begin{matrix}x_k^k\\
            {\mu}_k^k\end{matrix}\bigg)+p_k^k
            \end{align*}
            其中:$p_k^k=(p_x^k,p_{\mu}^k)^\mathrm{T} $为牛顿方向,满足方程
            \begin{align*}
            N(x^{k},{\mu}^k)p^k=-\nabla L(x^{k},{\mu}^k)
            \end{align*}
            即
            \begin{align*}
            \bigg[\begin{matrix}& G_k\quad & -A_k^\mathrm{T} \\
            & -A_k\quad & 0\end{matrix}\bigg]\bigg(\begin{matrix}p_x^k\\p_{\mu}^k\end{matrix}\bigg)=\bigg(\begin{matrix}& -\nabla f(x_k)+A(x_k)^\mathrm{T} {\mu}_k\\& h(x_k)\end{matrix}\bigg)
            \end{align*}
            其中:$G_k=G(x^k,{\mu}^k)$。如果下面条件成立,那么上述方程的系数矩阵非奇异,并且该方程有唯一解。
            \par
            假设约束函数$h$在$x^k$的Jacobi矩阵$A_k$行满秩;假设矩阵$g_k$在约束函数$h$的切空间$N(A_k)$上是正定的,即$\forall d \in N(A_k)/\{0\},d^\mathrm{T} w_kd>0$,则N-L方法具有局部二次收敛性质。但由于每一次迭代均求解非线性方程组,导致数值上的不稳定。鉴于这种不稳定性,所以将其转化为一个严格凸二次规划问题。转化的条件是原问题的解点$x^*$处最优化二阶充分条件成立,即对满足$A(x^*)^\mathrm{T} d=0$的任一非零向量$d$,有
            \begin{align*}
            d^\mathrm{T} G(x^*,{\mu}^*)d>0
            \end{align*}
            这时,当$\tau >0$充分小时,有
            \begin{align*}
            G(x^*,{\mu}^*)+\frac{1}{2\tau}A(x^*)^\mathrm{T} A(x^*)
            \end{align*}
            正定。考虑将方程组(\ref{eqSQP方法1})中的$G(x_k,{\mu}_k)$用一个正定矩阵来代替,记
            \begin{align*}
            B(x_k,{\mu}_k)=G(x^*,{\mu}^*)+\frac{1}{2\tau}A(x_k)^\mathrm{T} A(x_k)
            \end{align*}
            则当$(x_k,{\mu}_k) \to (x^*,{\mu}^*)$时,矩阵$B(x^*,{\mu}^*)$正定。注意到(\ref{eqSQP方法1})方程组的展开式为
            \begin{align*}
            G(x_k,{\mu}_k)d_k-A(x_k)^\mathrm{T} v_k=-\nabla f(x_k)+A(x_k)^\mathrm{T} {\mu}_k
            \end{align*}
            将上式变形为
            \begin{align*}
            [G(x_k,{\mu}_k)+\frac{1}{2\tau}A(x_k)^\mathrm{T} A(x_k)]d_k-A(x_k)^\mathrm{T} \left[ {\mu}_k+v_k+\frac{1}{2\tau}A(x_k)d_k \right]=-\nabla f(x_k)
            \end{align*}
            令
            \begin{align*}
            {\bar{\mu}}_k:={\mu}_k+v_k+\frac{1}{2\tau}A(x_k)d_k
            \end{align*}
            即得
            \begin{align*}
            B(x_k,{\mu}_k)d_k-A(x_k)^\mathrm{T} {\bar{\mu}}_k=-\nabla f(x_k)
            \end{align*}
            因此,方程组(\ref{eqSQP方法1})等价于
            \begin{align*}
            \bigg[\begin{matrix}& B(x_k,{\mu}_k)\quad & -A(x_k)^\mathrm{T} \\
            & -A(x_k)\quad & 0\end{matrix}\bigg]\bigg[\begin{matrix}d_k\\{\bar{\mu}}_k\end{matrix}\bigg]=-\bigg[\begin{matrix}\nabla f(x_k)\\h(x_k)\end{matrix}\bigg]
            \end{align*}
            进一步,可以将上述方程转化为严格凸二次规划
            \begin{align}
            \label{eqSQP方法2}
            & \mathop {\min}\limits_d \ q_k(d)=\frac{1}{2}d^\mathrm{T} B(x_k,{\mu}_k)d+\nabla f(x_k)^\mathrm{T} d\\
            & s.t.\quad h(x_k)+A(x_k)d=0 \notag
            \end{align}
            其中:$B(x_k,{\mu}_k)$是$n\times n$正定矩阵,$A(x_k)$是$m\times n$行满秩矩阵。
            \par
            上述凸二次规划的全局极小点等价于方程中的$d_k$。下面,给出纯等式约束优化问题的SQP算法流程:\\
            \textbf{step1.}初始化。
            $x_0 \in R^n,{\mu}_0 \in R^l,\rho,r \in (0,1),0 \leqslant \varepsilon \ll 1$,置$k:=0$。\\
            \textbf{step2.}计算$p(x_k,{\mu}_k)$的值。若$p(x_k,{\mu}_k) \leqslant \varepsilon$停止;否则转置step3。\\
            \textbf{step3.}求解(\ref{eqSQP方法2})式的凸二次规划,得到$\bar{\mu}_k,d_{k}$。并置
            \begin{align*}
             v_k=\bar{\mu}_k-{\mu}_k-\frac{1}{2\tau}A(x_k)d_{k}
            \end{align*}
            \textbf{step4.}若$p(x_k+d_k,{\mu}_k+v_k) \leqslant (1-r)p(x_k,{\mu}_k)$,
            则置${\alpha}_k:=1$,转到step6,否则转到step5。\\
            \textbf{step5.}令$m_k$是使下面的不等式成立的最小非负整数$m$
            \begin{align*}
             p(x_k+{\rho}^md_k,{\mu}_k+{\rho}^mv_k) \leqslant (1-r{\rho}^m)p(x_k,{\mu}_k)
            \end{align*}
            置${\alpha}_k={\rho}^{m_k}$。\\
            \textbf{step6.}令$x_{k+1}=x_k+{\alpha}_kd_k$,${\mu}_{k+1}={\mu}_k+{\alpha}_kv_k$,置$k:=k+1$,转到step2。
            \par
            不难发现,在上面的算法中,若${\alpha}_k < 1$,则必有
            \begin{align*}
             p(x_k+{\rho}^{m_{k-1}}d_k,{\mu}_k+{\rho}^{m_{k-1}}v_k) > (1-r{\rho}^{m_{k-1}})p(x_k,{\mu}_k)
            \end{align*}
            并且该算法具有全局收敛性:
            \par
            若SQP生成的序列$\{(x_k,{\mu}_k)\}$使得KKT矩阵的逆矩阵$N(x_k,{\mu}_k)^{-1}$一致有界,则$\{(x_k,{\mu}_k)\}$的任何聚点$(x^*,{\mu}^*)$都满足$p(x^*,{\mu}^*)=0$。特别地,$\{x_k\}$的任一聚点是原问题的KKT点。
            下面给出的是线性SQP的收敛速度。
            \par
            设SQP产生的序列$\{x_k\}$收敛到一个局部极小点$x^*$,若$f,h$在$x^*$附近二阶连续可微,Jacobi矩阵$A(x^*)=\nabla h(x)^\mathrm{T} $行满秩,且二阶最优化充分条件成立,则有
            \par
            (1)\ $\{{\mu}_k\} \to {\mu}^*$,其中,${\mu}^*$是等式约束问题的Lagrange乘子,且$\{(x_k,{\mu}_k)\}$是二阶收敛的,即
            \begin{align*}
             \|(x_{k+1}-x^*,{\mu}_{k+1}-{\mu}^*)\|=O(\|(x_{k}-x^*,{\mu}_{k}-{\mu}^*)\|^2)
            \end{align*}
            \par
            (2)序列$\{x_k\}$超线性收敛到$x^*$,且$t \in \mathbb{Z}$(正整数)
            \begin{align*}
             \|x_{k+1}-x^*\|=O(\|x_{k}-x^*\|\mathop {\Pi}\limits_{i=1}^t\|x_{k-i}-x^*\|)
            \end{align*}
        \subsubsection{一般约束优化的SQP算法}
            \par
            将前面的等式约束问题的SQP思想推广到一般形式的约束优化问题。在给定$(x_k,{\mu}_k,{\lambda}_k)$之后,将约束函数线性化,并且对$L(x,\lambda,\mu)$进行二次多项式近似,得到下列形式的二次规划子问题
            \begin{align}
            \label{eq:二次规划子问题}
            &\mathop {\min}\  \frac 12 d^\mathrm{T} G_kd+\nabla f(x_k)d\\
            &s.t.\left\{
            \begin{aligned}
            &h_i(x_k)+\nabla h_i(x_k)^\mathrm{T} d=0\quad i \in E\\
            &g_j(x_k)+\nabla g_j(x_k)^\mathrm{T} d \geqslant 0\quad j \in I
            \end{aligned}
            \right.\notag
            \end{align}
            其中:
            \begin{align*}
            &G_k=G(x_k,{\mu}_k,{\lambda}_k)={\nabla}_{xx}^2L(x_k,{\mu}_k,{\lambda}_k)\\
            &L(x,\mu,\lambda)=f(x)-\mathop{\sum}_i{\mu}_ih_i(x)-\mathop{\sum}_j{\lambda}_jg_j(x)
            \end{align*}
            于是$x_k$的校正步$d_k$以及新的乘子估计量${\mu}_{k+1},{\lambda}_{k+1}$可以分别定义为问题(\ref{eq:二次规划子问题})的最优解$d^*$和相应的拉格朗日乘子${\mu}^*,{\lambda}^*$。
            \par
            上述的二次规划子问题(\ref{eq:二次规划子问题})可能不存在可行点。为此,Powell引进一辅助变量$\xi$
            \begin{align*}
            &\mathop {\min} \ -\xi\\
            &s.t.\left\{
            \begin{aligned}
            &-{\xi}h_i(x_k)+\nabla h_i(x_k)^\mathrm{T} d=0\quad i \in E\\
            &-{\xi}g_k(x_k)+\nabla g_j(x_k)^\mathrm{T} d \geqslant 0\quad j \in u_k\\
            &g_k(x_k)+\nabla g_i(x_k)^\mathrm{T} d \geqslant 0\quad i \in v_k\\
            &-1 \leqslant \xi \leqslant 0
            \end{aligned}
            \right.
            \end{align*}
            其中:$u_k=\{i|g_i(x_k)<0,i\in I\}$,$v_k=\{i|g_i(x_k) \geqslant 0,i\in I\}$。
            \par
            注意到,在构建二次规划子问题(\ref{eq:二次规划子问题})时,需要计算$L(x,\lambda,\mu)$在迭代点$x_k$处的Hesse矩阵
            $G_k=G(x_k,{\mu}_k,{\lambda}_k)$。但其计算量巨大,为了克服这一缺陷,1976年,华裔数学家韩世平(Han)基于N-L方法提出了一种利用对称正定矩阵$B_k$替代$G_k$的SQP。另外,Wilson于1963年较早考虑N-L方法。Powell于1977年修正了Ham的方法,所以也称这种SQP为WHP方法。
            \par
            在迭代点$(x_k,{\mu}_k,{\lambda}_k)$处,WHP需要构造一个下列形式的二次规划子问题
            \begin{align}
            \label{eq:一般约束规划的二次规划子问题}
            &\mathop {\min} \ \frac 12 d^\mathrm{T} B_kd+\nabla f(x_k)^\mathrm{T} d\\
            &s.t.\left\{
            \begin{aligned}
            &h_i(x_k)+\nabla h_i(x_k)^\mathrm{T} d=0\quad i \in E\\
            &g_j(x_k)+\nabla g_j(x_k)^\mathrm{T} d \geqslant 0\quad j \in I
            \end{aligned}
            \right.
            \end{align}
            并且用此二次规划子问题的解$d_k$作为原问题变量$x_k$的搜索方向\footnote{注:搜索方向$d_k$是许多罚函数的下降方向,比如$\ell_1$罚函数。}。
            \par
            下面,给出WHP的算法步骤:\\
            \textbf{step1.}初始化。
            $x_0 \in R^n$,初始对称矩阵$B_0 \in R^{n\times n}$,容许误差$0<\xi \ll 1$,非负数列$\{\eta_k\}$,$\mathop {\sum}\limits_{k=0}^{\infty}{\eta_k}<{+\infty},\sigma > 0,\delta > 0$,置$k:=0$。\\
            \textbf{step2.}求解二次规划子问题,解得$d_k$。\\
            \textbf{step3.}若$\|d_{k}\|<\varepsilon$停止,输出$x_k$,否则转到step4。\\
            \textbf{step4.}计算$\ell_1$罚函数$p(x,\sigma)$
            \begin{align*}
            p(x,\sigma)=f(x)+\frac{1}{\sigma}\Big[\mathop{\sum}_i|h_i(x)|+\mathop{\sum}_j|[g_j(x)]_-|\Big]
            \end{align*}
            其中:罚权重$\sigma > 0,[g_j(x)]_-=\max\{0,-g_j(x)\}$。
            并按照某种线搜索规则确定步长${\alpha}_k \in (0,\delta ]$使得
            \begin{align*}
             p(x_k+{\alpha}_kd_k,\sigma) \leqslant \mathop{\min}\limits_{\alpha \in (0,\delta]}p(x_k+{\alpha}d_k,\sigma)+{\eta}_k
            \end{align*}
            \textbf{step5.}置$x_{k+ 1}:=x_k+{\alpha}_kd_k$,更新$B_k$为$B_{k+1}$,置$k:=k+1$,转到step2。
            \par
            下面,我们给出WHP的全局收敛性。
            设$f,h_i,g_j$是连续可微的,且$\exists 0<m \leqslant M$,使对称正定矩阵$B_k$满足
            \begin{align*}
             m {\|d\|}^2\leqslant d^\mathrm{T} B_kd \leqslant M\|d\|^2\quad (\forall d \in R^n)
            \end{align*}
            \par
            若罚权重$\sigma > 0$和二次规划子问题的拉格朗日乘子向量${\mu}_{k+1},{\lambda}_{k+1} \geqslant 0$满足
            \begin{align*}
             \sigma \max \{\|{\lambda}_{k+1}\|_{\infty},\|{\mu}_{k+1}\|_{\infty}\}\leqslant 1 \quad (\forall k)
            \end{align*}
            则WHP产生的序列$\{x_k\}$的任何聚点都是原问题的KKT点。在上述WHP中有两个存留的问题:\ding{172}关于$B_{k+1}$的计算。\ding{173}关于${\alpha}_k$的求解。下面介绍求解$B_{k+1}$的Powell方法和增广拉格朗日函数法,以及求解$\alpha_k$的价值函数法。
            \paragraph{Powell的方法}
            $B_{k+1}$的计算一般是用拟牛顿修正公式逐步迭代产生,我们希望$B_{k+1}$是Lagrange函数Hesse矩阵$G \triangleq {\nabla}^2_{xx}L$的近似。我们令
            \begin{align*}
             &s_k=x_{k+1}-x_k\\
             &y_k={\nabla}_xL(x_{k+1},{\mu}_{k+1})-{\nabla}_xL(x_{k},{\mu}_{k+1})
            \end{align*}
            因为BFGS校正公式要求$s_k,y_k$满足曲率条件。$s_k^\mathrm{T} y_k>0$,但上式确定的$s_k,y_k$可能不满足这一条件。为此,有必要对$y_k$进行修正。Powell于1978年建议用下式对$y_k$进行修正
            \begin{align*}
            {\bar{y}}_k = \left\{
            \begin{aligned}
            &y_k \quad s_k^\mathrm{T} y_k \geqslant 0.2s_k^\mathrm{T} B_ks_k\\
            &\theta_k y_k+(1-\theta_k)B_ks_k\quad \text{其它}
            \end{aligned}
            \right.
            \end{align*}
            其中:${\theta}_k=\frac{0.8s_k^\mathrm{T} B_ks_k}{s_k^\mathrm{T} B_ks_k-s_k^\mathrm{T} y_k}$。
            \par
            这种选取${\bar{y}}_k$的基本思想是:利用$y_k$和$B_ks_k$的凸组合构造一可以用来修正矩阵的向量。由于$B_ks_k$可理解为$y_k$的一种近似估计,且满足(因为$B_k$正定)
            \begin{align*}
             s_k^\mathrm{T} (B_ks_k)>0
            \end{align*}
            故利用$y_k$和$B_ks_k$的凸组合是一种很自然的选择。于是,矩阵$B_k$的约束BFGS校正公式为
            \begin{align*}
             B_{k+1}=B_k-\frac{B_ks_ks_k^\mathrm{T} B_k^\mathrm{T} }{s_k^\mathrm{T} B_ks_k}+\frac{{\bar{y}}_k{\bar{y}}_k^\mathrm{T} }{s_k^\mathrm{T} {\bar{y}}_k}
            \end{align*}
            由${\bar{y}}_k$的定义,不难验证$s_k^\mathrm{T} {\bar{y}}_k>0$。
            \paragraph{基于增广拉格朗日函数的选择}
            另一种选择$B_{k+1}$的方法是基于增广拉格朗日函数。考虑下面的增广拉格朗日函数
            \begin{align*}
             L_A(x,\lambda,\mu)=f(x)-{\lambda}^\mathrm{T} h(x)+\frac{1}{2\mu}\|h\|_2^2
            \end{align*}
            其中:罚权重$\mu > 0$。在局部极小点$(x^*,{\lambda}^*)$处,根据$h(x^*)=0$,可知增广拉格朗日函数的Hesse矩阵
            \begin{align*}
             {\nabla}^2_{xx}L_A(x^*,{\lambda}^*,{\mu})={\nabla}^2_{xx}L(x^*,{\lambda}^*)+{\mu}^{-1}A(x^*)^\mathrm{T} A(x^*)
            \end{align*}
            其中:等式右边第二项为曲率,对应着约束函数在$x^*$点的法向量张成的空间$\mathcal{R}(A^\mathrm{T} )$。对于纯等式约束的优化问题,若约束函数在$x^*$处的Jacobi矩阵$A(x^*)$行满秩,并且二阶最优化充分条件成立,则存在某个阈值$\bar{\mu}$,使得$\forall \mu \in (0,\bar{\mu}],{\nabla}^2_{xx}L_A(x_k,{\lambda}_k,{\mu})$是正定的。于是,$G(x_k,{\lambda}_k)$可以取成正定矩阵${\nabla}^2_{xx}L_A(x_k,{\lambda}_k,{\mu})$或者对${\nabla}^2_{xx}L_A$进行拟牛顿法近似的校正矩阵$B_k$。
            \paragraph{步长${\alpha}_k$的选取}
            为了保证SQP的全局收敛性,通常借助某价值函数来确定搜索步长${\alpha}_k$。如:目标函数$f$、罚函数$p$、增广拉格朗日函数$L_A$等都可以作为价值函数。
            \par
            $\ell_1$价值函数:
            对于一般的约束优化问题,可以考虑相应的二次规划子问题(\ref{eq:一般约束规划的二次规划子问题})
            % \begin{align}
            % \label{eq:一般约束规划的二次规划子问题}
            % &\mathop {\min} \ \frac 12 d^\mathrm{T} B_kd+\nabla f(x_k)^\mathrm{T} d\\
            % &s.t.\left\{
            % \begin{aligned}
            % &h_i(x_k)+\nabla h_i(x_k)^\mathrm{T} d>0\quad i \in E\\
            % &g_j(x_k)+\nabla g_j(x_k)^\mathrm{T} d \geqslant 0\quad j \in I
            % \end{aligned}
            % \right. \notag
            % \end{align}
            并且将罚函数写为如下的$\ell_1$罚函数($\ell_1$价值函数的形式)
            \begin{align}
            \label{eq:价值函数的形式}
            p(x,\sigma) = f(x)+ \frac {1}{\sigma} \left( \|{g(x)}_{-}\|_1+\|h(x)\|_1 \right)
            \end{align}
            设$d_k {\lambda}^{k+1} \geqslant 0,{\mu}^{k+1}$为问题(\ref{eq:一般约束规划的二次规划子问题})的最优解和拉格朗日乘子向量,则$p(x,\sigma)$沿$d_k$的方向导数满足
            \begin{align*}
            D(p(x_k,\sigma),d_k) \leqslant -(d_k)^\mathrm{T} B_kd_k-({\sigma}^{-1}-\|{\lambda}_{k+1}\|_{\infty})\|(g(x))_{+}\|_1-({\sigma}^{-1}-\|{\mu}_{k+1}\|_{\infty})\|h_k\|
            \end{align*}
            其中:$(g_k)_{+}=\max \{0,-g(x_k)\}$。
            \par
            当然我们还有一些其它的价值函数可以选择,比如增广拉格朗日价值函数等\footnote{可以参考《数学规划》黄红选P302或者《最优化方法与Matlab程序设计》P231。}。
            \par
            下面给出一般约束优化问题的SQP算法流程\\
            \textbf{step1.}初始化。给定初始点对$(x_0,\lambda_0,\mu_0)$,对称正定矩阵$B_0$,计算$f_0 = f(x_0)$,$\nabla f_0 = \nabla f(x_0)$,$g_0 = g(x_0)$,$h_0 = h(x_0)$,$A_0^\mathrm{T} = (\nabla g(x^0),\nabla h(x^0))$,选择参数$\eta \in (0,1/2)$,$\tau \in (0,1)$,容许误差$\epsilon_1,\epsilon_2>0$,置$k:=0$。\\
            \textbf{step2.}计算改进方向。求解二次规划自问题,得到变量$x_k$的改进方向$d_k$。\\
            \textbf{step3.}收敛性检验。若$||d_k||_1 \leqslant \epsilon_1$,并且$||(g_k)_-||_1+||h_k||_1 \leqslant \epsilon_2$成立,则得到约束优化问题的一个近似KKT点$(x_k,\lambda_k,\mu_k)$,算法终止;否则,转到step4。\\
            \textbf{step4.}确定罚因子。对于某种价值函数$\phi(x,\sigma)$,选择罚因子$\sigma_k$,使得$d_k$为该函数在$x_k$点的下降方向。\\
            \textbf{step5.}步长选择。在序列$1,\tau,\tau^2,\dots$中,选择最大的项作为$\alpha_k$,使得
            \begin{align*}
            \phi(x_k+\alpha_k d_k,\sigma_k) \leqslant \phi(x_k,\sigma_k)+\eta \alpha_k D (\phi(x_k,\sigma_k),d_k)
            \end{align*}
            \textbf{step6.}改进迭代点。令$x_{k+1} = x_{k}+\alpha_k d_k$,并且计算
            \begin{align*}
            & f_{k+1} = f(x_{k+1})\\
            & \nabla f_{k+1} = \nabla f(x_{k+1})\\
            & g_{k+1} = g(x_{k+1})\\
            & h_{k+1} = h(x_{k+1})\\
            & A_{k+1}^{\mathrm{T}} = (\nabla g(x_{k+1}),\nabla h(x_{k+1}))
            \end{align*}
            以及最小二乘乘子
            \begin{align*}
            \begin{pmatrix}
            \lambda_{k+1}\\
            \mu_{k+1}
            \end{pmatrix}
            =
            (A_{k+1}A_{k+1}^\mathrm{T})^{-1}A_{k+1}\nabla f_{k+1}
            \end{align*}
            \textbf{step7.}校正Hesse矩阵。令
            \begin{align*}
            s_k = \alpha_k d_k,\quad y_k = \nabla_x L(x_{k+1},\lambda_{k+1},\mu_{k+1}) - \nabla_x L(x_k,\lambda_{k+1},\mu_{k+1})
            \end{align*}
            利用约束牛顿公式修正矩阵$B_k$,生成新的对称正定矩阵$B_{k+1}$,令$k:=k+1$,返回step2。
            \par
            在上述算法中,我们隐设了矩阵$A_k$是行满秩的。如果这个条件不成立,那么在计算最小二乘乘子时,就需要使用计算矩阵广义逆的技巧。最后,值得指出的是,对于无约束优化问题,如果$x^*$是驻点,并且目标函数$f(x)$在该点满足二阶最优性充分条件,即Hesse矩阵正定,那么只要迭代点列$\{x_k\}$收敛到$x^*$,并且搜索方向列$\{d_k\}$满足
            \begin{align*}
            \lim _{k\to \infty} \frac{||x_k+d_k-x^*||}{||x_k-x^*||} = 0
            \end{align*}
            对于充分大的$k$,必然有
            \begin{align*}
            f(x_k+d_k) < f(x_k)
            \end{align*}
            我们把这样的$d_k$称为超线性收敛步。
\section{MATLAB应用实例}
    \par
    MATLAB通过fmincon函数来求解约束优化问题,其调用格式为
    \par
    [x,fval,exitflag,output,lambda,grad,hessian]=fmincon(fun,x0,A,b,Aeq,beq,lb,ub,nonlan,options)\\
    其中:fun为目标函数;x0为初始点;A,b为线性不等式约束$Ax \leqslant b$;Aeq,beq为线性等式约束Aeq x=beq;lb,ub为自变量x的上下限,$lb \leqslant x \leqslant ub$;nonlcon为非线性约束条件;options为结构体参数;lambda为最优点$x$处的拉格朗日乘子;grad为最优点$x$处的梯度;hessian为最优点$x$处的Hesse矩阵;
    \par
    我们用fmincon函数来求解如下非线性约束优化问题
    \begin{align*}
    &\mathop {\min}\  f(x)=x_1^4-4x_1-8x_2+15\\
    &s.t.\left\{
    \begin{aligned}
    &9-x_1^2-x_2^2 \leqslant 0\\
    &2x_1+3x_2 \leqslant 2\\
    &x_2-x_1 \leqslant 5
    \end{aligned}
    \right.
    \end{align*}
    其求解程序为
    \begin{lstlisting}[language=Matlab]
    fun = @(x)x(1)^4-4x(1)-8x(2)+15;
    x0 = [1,2];
    A = [2,3;1,-1];
    b = [2;5];
    Aeq = [];
    beq = [];
    Lb = [];
    Ub = [];
    function[c, ceq] = ConFun(x)
        c = 9-x(1)^2-x(2)^2;
        ceq = [];
    end
    nonlcon = @ConFun
    options = optimoptions('fmincon', 'Display', 'iter', 'Algorithm', 'sqp');
    x = fmincon(fun,x0,A,b,Aeq,beq,lb,ub,nonlcon,option)
    \end{lstlisting}











% \end{document}
