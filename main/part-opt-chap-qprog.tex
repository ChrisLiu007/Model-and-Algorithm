% % \part{优化模型}
% % \chapter{二次规划}

% \documentclass[UTF8]{ctexbook}

% \ctexset{
%     part/number = \chinese{part}
% }
% \usepackage{amsmath}% ams 数学公式
% \usepackage{mathtools}% ams 数学公式
% \usepackage{amsfonts}% ams 数学字体
% \usepackage{amssymb,latexsym}% ams 数学符号与LaTeX数学符号
% \usepackage{mathrsfs}% 花式符号
% \usepackage{ntheorem}%定理、定义、证明
%   \theoremstyle{nonumberplain}
%   \theoremheaderfont{\bfseries}
%   \theorembodyfont{\normalfont}
%   \theoremsymbol{$\square$}
%   \newtheorem{Proof}{\hskip 2em 证明}
%   \newtheorem{theorem}{\hspace{2em}定理}[chapter]
%   \newtheorem{definition}{\hspace{2em}定义}[chapter] % 如果没有章, 只有节, 把上面的[chapter]改成[section]
%   \newtheorem{axiom}[definition]{\hspace{2em}公理}
%   \newtheorem{lemma}[definition]{\hspace{2em}引理}
%   \newtheorem{proposition}[definition]{\hspace{2em}命题}
%   \newtheorem{corollary}[definition]{\hspace{2em}推论}
%   \newtheorem{remark}{\hspace{2em}注}[chapter] %类似地定义其他“题头”. 这里“注”的编号与定义、定理等是分开的

% \usepackage{enumerate}%itemiz环境。\begin{enumerate}[step 1]
% \usepackage{cite}%参考文献
%     \bibliographystyle{plain}
% \usepackage{extarrows}% 带参数的箭头
% \usepackage{hyperref}% 超链接
% %\usepackage[CJKbookmarks, colorlinks, bookmarksnumbered=true,pdfstartview=FitH,linkcolor=black,citecolor=black]{hyperref}%超链接的格式设置
% \hypersetup{
%     colorlinks=false,% 去掉超链接颜色
%     pdfborder=0 0 0% 取消超链接的边框
% }
% \usepackage{graphicx}% 图片管理
% \usepackage{caption}
% \usepackage{subcaption}%并排的图各有标题
% \graphicspath{{images/}}% 设置图片搜索路径
% \usepackage{float,varwidth}% 浮动体
% \usepackage{booktabs}% 三线表
% \usepackage{fancyhdr}% 页眉设置
% \usepackage{xcolor}% 颜色宏包
% \usepackage{colortbl}% 彩色表格
% \usepackage{listings}% 代码高亮
% \usepackage{caption}% 对标题进行控制,如让\caption标题的字体缩小一号,同时数字标签使用粗体可以用:\usepackage[font=small,labelfont=bf]{caption}
% \usepackage{xfrac,upgreek}%分别是行间公式如a/b的形式(将原来的命令\frac改成\sfrac)和希腊字体的宏包的
% \usepackage{mathtools}%lgathered和rgathered环境把公式向左向右对齐
% \usepackage{tabularx}%提供自动延伸的表列,(X列格式说明符),文字过长时可以自动转行
% \usepackage{longtable}%长表格
% \usepackage{enumitem}%enumerate宏包的升级
% \usepackage{harpoon}%数学公式的矢量
% \usepackage{bookmark}%目录的书签
% \usepackage{pifont}%给数字加上圈。然后在正文输入\ding{172}~\ding{211}得到相应数字,要是要①就输入:\ding{172}②就输:\ding{173}
% \renewcommand{\headwidth}{\textwidth}%图片并排,这个要列在所有宏包的后面
% \makeatletter
% \newcommand{\rmnum}[1]{\romannumeral #1}
% \newcommand{\Rmnum}[1]{\expandafter\@slowromancap\romannumeral #1@}
% \makeatother
% \definecolor{codegreen}{rgb}{0,0.6,0}
% \definecolor{codegray}{rgb}{0.5,0.5,0.5}
% \definecolor{codepurple}{rgb}{0.58,0,0.82}
% \definecolor{backcolour}{rgb}{0.95,0.95,0.92}
% \lstset{
%     commentstyle=\color{codegreen},
%     keywordstyle=\color{magenta},
%     numberstyle=\tiny\color{codegray},
%     stringstyle=\color{codepurple},
%     basicstyle=\footnotesize,
%     breakatwhitespace=false,% 断行只在空格处
%     breaklines=true,% 自动断行
%     captionpos=b,% 标题位置
%     keepspaces=true,
%     numbers=left,
%     numbersep=5pt,
%     showspaces=false,
%     showstringspaces=false,
%     showtabs=false,% 显示
%     tabsize=2% TAB 被当作两个空格
% }
% \topmargin=0pt\oddsidemargin=0pt\evensidemargin=0pt
% \textwidth=16.5cm\textheight=23cm\raggedbottom%我这么设置是为了缩小页边距,满足有的文字无法转行
% \pagestyle{headings}%页眉为章节标题,无页脚
% \setlength{\abovecaptionskip}{4pt}
% \setlength{\belowcaptionskip}{-8pt}%图片表格的前后距离设置
% \CTEXsetup[format={\zihao{-3}\raggedright\bfseries}]{section}%设置节的格式

% \begin{document}
% \part{优化模型}
\chapter{二次规划}

\section{问题的引入与分析}
    \label{sec:二次规划引例}
    \par
    考虑如下组合投资问题:设有$n$支股票,第$i$支股票的收益率为$r_i$(当然,就时间序列而言,此需要平稳性前提)。我们假设$[0,T]$时间内$r_i(t)$独立同分布于$N({\mu}_i,{\sigma}_i^2)$,则
    \begin{align*}
    &{\mu}_i = E(r_i)\\
    &{\sigma}_i^2=E[(r_i-{\mu}_i)^2]
    \end{align*}
    现在,我们决定将现有资产中的$x_i$比例投资股票$i$。
    问:如何的组合情况$x=(x_1,x_2,\ldots,x_n)^\mathrm{T} $,能使投资的收益最大风险最小。
    \par
    解:目标1,我们要求股票收益最大。我们将收益定义为
    \begin{align*}
    R=x^\mathrm{T} r=\mathop{\sum}\limits_{i=1}^n x_ir_i
    \end{align*}
    \par
    目标2,投资风险最小。风险的度量有许多种,这里,我们选取最常用的方差作为度量。投资组合$R=x^\mathrm{T} r$的方差为
    \begin{align*}
    E(R-E(R))^2&=E \left( \mathop{\sum}\limits_{i} x_i(r_i-{\mu}_i) \right) ^2\\
    &=\mathop{\sum}\limits_{i}\mathop{\sum}\limits_{j}x_ix_j{\sigma}_{ij}\\
    &=x^\mathrm{T} Qx
    \end{align*}
    其中:${Q}=(cor(r_ir_j))_{n\times n}=(\sigma_{ij})_{n\times n}$。
    \par
    于是,得到的优化模型为
    \begin{align*}
    &\mathop {\min}\limits_x\  x^\mathrm{T} Qx\\
    &\mathop {\max}\limits_x\  x^\mathrm{T} r\\
    &s.t.\left\{
    \begin{aligned}
    &x^\mathrm{T} e=1\\
    &x \geqslant 0\\
    &i=1,2,\ldots,n
    \end{aligned}
    \right.
    \end{align*}
    \par
    上面的模型有2个目标,而且这2个目标相互冲突:不能同时达到最优。这是一个多目标规划问题,具体处理方法在多目标章节介绍。下面,我们对其进行一定的处理。
    \par
    (1)要求收益在不小于一定值的条件下风险最小,则有
    \begin{align*}
    &\mathop {\min}\limits_x \ x^\mathrm{T} Qx\\
    &s.t.\left\{
    \begin{aligned}
    &x^\mathrm{T} e=1\\
    & 0 \leqslant x \leqslant 1\\
    &x^\mathrm{T} r > b
    \end{aligned}
    \right.
    \end{align*}
    其中:$x \in R^n$,$Q \in R^{n\times n}$,$e \in R^n$,$b \in R$,$r \in R^n$,$b$为最低收益常数。
    \par
    (2)也可以将多目标合并为单目标,设目标权重为$k$,则有
    \begin{align*}
    &\mathop {\min}\limits_x\  x^\mathrm{T} Qx-kx^\mathrm{T} r\\
    &s.t.\left\{
    \begin{aligned}
    &x^\mathrm{T} e=1\\
    &0 \leqslant x_i \leqslant 1\\
    & i = 1,2,\dots,n
    \end{aligned}
    \right.
    \end{align*}
    \par
    用MATLAB解组合投资的二次规划问题。MATLAB采用以下三种算法来求解二次规划:
    \begin{enumerate}
    \item 内点凸包算法:具有任何约束组合的凸包问题;
    \item 信赖域反射算法:求解边界约束或线性等式约束问题;
    \item 动态序列算法:求解任何约束组合的问题。
    \end{enumerate}
    \par
    设组合投资的二次规划模型如下
    \begin{align*}
    &\mathop {\min}\limits_{x \in R^n}\  f(x)=x^\mathrm{T} Qx\\
    &s.t.\left\{
    \begin{aligned}
    &\mathop {\sum}\limits_{i=1}^n x_ir_i \geqslant r\\
    &\mathop {\sum}\limits_{i=1}^n x_i = 1\\
    &0 \leqslant x_i \leqslant 1\\
    &i=1,2,\ldots,n
    \end{aligned}
    \right.
    \end{align*}
    并且假设现在有225支股票,即$n=225$。其求解程序如下
    \begin{lstlisting}[language = Matlab]
    %% 组合投资二次规划问题
    %加载数据
    load('port5.mat','Correlation','stdDev_return','mean_return')
    %计算相关向量矩阵Q
    Covariance = Correlation .* (stdDev_return * stdDev_return');
    nAssets = numel(mean_return);
    r = 0.002;%获利下限
    Aeq = ones(1,nAssets);
    beq = 1;
    Aineq = -mean_return';
    bineq = -r;
    lb = zeros(nAssets,1);
    ub = ones(nAssets,1);
    c = zeros(nAssets,1);
    options = optimoptions('quadprog','Algorithm','interior-point-convex');
    options = optimoptions(options,'Display','iter','TolFun',1e-10);
    [x1,fval1] = quadprog(Covariance,c,Aineq,bineq,Aeq,beq,lb,ub,[],options);
    plotPortfDemoStandardModel(x1)
    % 分组投资225-Asset Problem with Group Constraints
    % 新的投资组合要求:
    % sum_1_75 xi>=0.3;sum_76_150 xi>=0.3;sum_151_225 xi>=0.3
    Groups = blkdiag(ones(1,nAssets/3),ones(1,nAssets/3),ones(1,nAssets/3));
    Aineq = [Aineq; -Groups];         % convert to <= constraint
    bineq = [bineq; -0.3*ones(3,1)];  % by changing signs
    [x2,fval2] = quadprog(Covariance,c,Aineq,bineq,Aeq,beq,lb,ub,[],options);
    % Plot results, superimposed to results from previous problem.
    plotPortfDemoGroupModel(x1,x2);
    %% 1000支股票
    % 为了展示quadprog的内点算法能解决更大的问题,我们使用一个随机生成1000的数据集。
    % 我们生成一个随机相关矩阵(对称半正定,对角线)。
    % Reset random stream for reproducibility.
    rng(0,'twister');
    nAssets = 1000;
    % Generate means of returns between -0.1 and 0.4.
    a = -0.1;
    b = 0.4;
    mean_return = a + (b-a).*rand(nAssets,1);
    % Generate standard deviations of returns between 0.08 and 0.6.
    a = 0.08;
    b = 0.6;
    stdDev_return = a + (b-a).*rand(nAssets,1);
    % Correlation matrix, generated using Correlation = gallery('randcorr',nAssets).
    % (Generating a correlation matrix of this size takes a while, so we load
    % a pre-generated one instead.)
    load('correlationMatrixDemo.mat','Correlation');
    % Calculate covariance matrix from correlation matrix.
    Covariance = Correlation .* (stdDev_return * stdDev_return');
    % Define and Solve Randomly Generated 1000-Asset Problem
    % We now define the standard QP problem (no group constraints here) and solve.
    r = 0.15;
    Aeq = ones(1,nAssets);
    beq = 1;
    Aineq = -mean_return';
    bineq = -r;
    lb = zeros(nAssets,1);
    ub = ones(nAssets,1);
    c = zeros(nAssets,1);
    x3 = quadprog(Covariance,c,Aineq,bineq,Aeq,beq,lb,ub,[],options);
    \end{lstlisting}
\section{模型的规范化及基本理论}
    \subsection{规范化}
        \par
        根据上面的例子,我们写出二次规划的一般形式
        \begin{align*}
        &\mathop {\min}\limits_{x\in R^n}\ f(x)=\frac 12 x^\mathrm{T} Qx+c^\mathrm{T} x\\
        &s.t.\left\{
        \begin{aligned}
        &A_1x \leqslant b_1\\
        &A_2x = b_2
        \end{aligned}
        \right.
        \end{align*}
        或者写为
        \begin{align*}
        &\mathop {\min}\limits_{x}\ f(x)=\frac 12 \mathop{\sum}\limits_{i}\mathop{\sum}\limits_{j}x_ix_j{Q}_{ij}+\mathop {\sum}\limits_{i=1}^n c_ix_i\\
        &s.t.\left\{
        \begin{aligned}
        &a_i^\mathrm{T} x \leqslant b_i \quad i \in I\\
        & a_j^\mathrm{T} x = b_j \quad j \in E
        \end{aligned}
        \right.
        \end{align*}
        其中:$Q=Q^\mathrm{T}  \in R^{n \times n}$,$x \in R^n$,$I=\{1,2,\ldots,m\}$,$E=\{1,2,\ldots,l\}$,$|I|=m$,$|E|=l$,$A_1 \in R^{|I|\times n}$,$A_2 \in R^{|E|\times n}$。由凸函数判定定理,我们有:如果目标函数$f$的Hesse矩阵$Q$是正定的,则该二次规划称为凸二次规划问题。
    \subsection{最优化条件}
        \paragraph{必要条件}
        设$x^*$为二次规划的一个局部极小点,则存在乘子${\lambda}^* \in R^{|I|}_{+}$,${\mu}^* \in R^{|E|}$,有
        \begin{align*}
        &Qx^*+c=A_1^\mathrm{T} {\lambda}^*+A_2^\mathrm{T} {\mu}^*\\
        &(A_1x^*-b_1)^\mathrm{T} {\lambda}^*=0\\
        &{\lambda}^* \geqslant 0
        \end{align*}
        且对一切满足
        \begin{align*}
        d^\mathrm{T} a_i=0\quad i \in E \cup I(x^*)
        \end{align*}
        的$d$\ ($d\in R^n$),有
        \begin{align*}
        d^\mathrm{T} Qd \geqslant 0
        \end{align*}
        其中:$E=\{1,2,\ldots,l\}$,$I(x^*)=\{i|a_i^\mathrm{T} x^*=b_i,i \in I\}$。
        \paragraph{充分条件}
        设$x^*$为二次规划的一个KKT点,${\lambda}^*{\mu}^*$为Lagrange乘子。如果对于一切满足
        \begin{align}
          \label{充分条件}
          \left\{
            \begin{aligned}
                &d^\mathrm{T} a_i = 0 \quad i\in E\\
                &d^\mathrm{T} a_i \geqslant 0 \quad i\in I(x^*)\\
                &{d}^\mathrm{T} a_i = 0 \quad i\in I(x^*),{\lambda}^*_i>0,{\mu}^*_i>0
            \end{aligned}
          \right.
        \end{align}
        的非零向量$d$,都有
        \begin{align*}
        d^\mathrm{T} Qd>0
        \end{align*}
        则$x^*$为二次规划的拒不严格极小点。
        \paragraph{充分必要条件}
        设$x^*$为二次规划的可行点,则$x^*$是局部极小点,当且仅当存在${\lambda}^*{\mu}^*$,使得
        \begin{align*}
                &Qx^*+c = A_1^\mathrm{T} {\lambda}^*+A_2^\mathrm{T} {\mu}^*\\
                &(A_1x^*-b_1)^\mathrm{T} {\lambda}^*=0\\
                &{\lambda}^* \in R^{|I|}_{+},\ {\mu}^* \in R^{|E|}
        \end{align*}
        成立,且对一切满足(\ref{充分条件})式的向量$d$有
        \begin{align*}
        d^\mathrm{T} Qd \geqslant 0
        \end{align*}
        注:对凸二次规划而言,$x^*$为全局极小点,当且仅当$x^*$为局部极小点,$x^*$为KKT点。
    \subsection{对偶问题}
        \par
        二次规划的对偶理论最先由J.B.Dennis和W.S.Dorn等人于1959年和1960年给出。令
        \begin{align*}
        A=\begin{pmatrix} A_1 \\ A_2\end{pmatrix}\quad b=\begin{pmatrix} b_1\\b_2 \end{pmatrix} \quad w=\begin{pmatrix}\lambda \\\mu \end{pmatrix}
        \end{align*}
        则二次规划的拉格朗日函数为
        \begin{align*}
         L(x,\lambda,\mu)&=\frac{1}{2}x^\mathrm{T} Qx+c^\mathrm{T} x-{\lambda}^\mathrm{T} (A_1x-b_1)-{\mu}^\mathrm{T} (A_2x-b_2)\\
         &=\frac{1}{2}x^\mathrm{T} Qx+x^\mathrm{T} (c-A^\mathrm{T} w)+b^\mathrm{T} w
        \end{align*}
        \par
        若$Q^\mathrm{T} =Q\succ 0$,则Lagrange对偶函数$\theta(\lambda,\mu)=\mathop {\inf}\limits_{x \in R^n}L(x,\lambda,\mu)$存在,并且对应的最小点$x^*$满足
        \begin{align*}
         {\nabla}_xL=Qx+c-A^\mathrm{T} w=0
        \end{align*}
        于是
        \begin{align*}
         x=-Q^{-1}(c-A^\mathrm{T} w)
        \end{align*}
        将其代入Lagrange函数,有
        \begin{align*}
        & \mathop{\max}\limits_{\lambda \geqslant 0}\  \theta(\lambda,\mu)=-\frac 12(c-A^\mathrm{T} w)^\mathrm{T} Q^{-1}(c-A^\mathrm{T} w)+b^\mathrm{T} w
        \end{align*}
        即
        \begin{align}
        \label{对偶问题的拉格朗日函数1}
        & \mathop{\max}\limits_{\lambda \geqslant 0}\  \theta(\lambda,\mu)=-\frac 12w^\mathrm{T} (AQ^{-1}A^\mathrm{T} )w+w^\mathrm{T} (b+AQ^{-1}c)-\frac 12 c^\mathrm{T} Q^{-1}c
        \end{align}
        其中:$w=(\lambda,\mu)^\mathrm{T} $。
        \par
        如果令$y=c-A^\mathrm{T} w$,则有
        \begin{align}
          \label{对偶问题的拉格朗日函数2}
          & \mathop{\max}\limits_{y,w}\  \theta(\lambda,\mu)=-\frac 12y^\mathrm{T} Q^{-1}y+b^\mathrm{T} w\\
          & s.t.\left\{
            \begin{aligned}
                &A^\mathrm{T} w+y=c\\
                &w=(\lambda,\mu)^\mathrm{T} \\
                &\lambda \geqslant 0
            \end{aligned}
          \right.\notag
        \end{align}
        上面的式(\ref{对偶问题的拉格朗日函数1})(\ref{对偶问题的拉格朗日函数2})为二次规划对偶的两种形式。值得一提的是,上面要求二次规划的$Q$为正定,即原二次规划为凸。为了得到适用性更广泛的对偶形式,令$y=Qz$,则有
        \begin{align*}
          & \mathop{\max}\limits_{y,w}\ \theta(\lambda,\mu)=-\frac 12z^\mathrm{T} Qz+b^\mathrm{T} w\\
          & s.t.\left\{
            \begin{aligned}
                &A^\mathrm{T} w+Qz=c\\
                &w=(\lambda,\mu)^\mathrm{T} \\
                &\lambda \geqslant 0
            \end{aligned}
          \right.
        \end{align*}
        此时,矩阵$Q$可以是半正定的。
        \par
        对于原问题和对偶问题,有
        \begin{align*}
        f(x)-\theta(\lambda,\mu)&=c^\mathrm{T} x-b^\mathrm{T} w+\frac{1}{2}x^\mathrm{T} Qx+\frac{1}{2}y^\mathrm{T} Q^{-1}y\\
        &={\lambda}^\mathrm{T} (A_1x-b_1)+\frac{1}{2}(x^\mathrm{T} Qx+y^\mathrm{T} Q^{-1}y+2x^\mathrm{T} y) \geqslant 0
        \end{align*}
        当且仅当$A{\lambda}^\mathrm{T} (A_1x-b_1)=0,x=-Q^{-1}y$时,$f(x)-\theta(\lambda,\mu)=0$。对于严格凸二次规划,记可行域为$S$,则$x^* \in S$为最优解,当且仅当$\exists {\lambda}^* \geqslant 0,{\mu}^*$使得$(x^*,{\lambda}^*,{\mu}^*)$为Lagrange函数$L(x,\lambda,\mu)$的鞍点,即对$\forall x \in S,\forall \lambda \geqslant 0$,有
        \begin{align*}
        L(x^*,{\lambda},{\mu}) \leqslant L(x^*,{\lambda}^*,{\mu}^*) \leqslant L(x,{\lambda}^*,{\mu}^*)
        \end{align*}
\section{最优化算法}
    \subsection{Lagrange方法}
        \par
        考虑只含等式约束的二次规划问题
        \begin{align*}
          & \mathop{\min}\limits_{x}\  \frac 12x^\mathrm{T} Qx+c^\mathrm{T} x\\
          & s.t.\quad Ax=b
        \end{align*}
        其中:$c,x \in R^n$,$Q \in R^{n \times n}$为对称矩阵,$A \in R^{m \times n}$,$rank(A)=m$,$b \in R^m$。首先,写出Lagrange函数
        \begin{align*}
         L(x,\lambda)=\frac{1}{2}x^\mathrm{T} Qx+c^\mathrm{T} x-{\lambda}^\mathrm{T} (Ax-b)
        \end{align*}
        令
        \begin{align*}
         &{\nabla}_xL(x,\lambda)=0\\
         &{\nabla}_{\lambda}L(x,\lambda)=0
        \end{align*}
        得到方程组
        \begin{align*}
         &Qx+c-A^\mathrm{T} \lambda=0\\
         &-Ax+b=0
        \end{align*}
        将上述方程写为矩阵形式
        \begin{align}
        \label{最优化算法1}
        \begin{bmatrix} Q & -A^\mathrm{T}  \\
        -A & 0 \end{bmatrix}\begin{bmatrix} x\\\lambda \end{bmatrix}=\begin{bmatrix}-c \\ -b \end{bmatrix}
        \end{align}
        系数矩阵
        \begin{math}
        \left(
        \begin{smallmatrix} Q & -A^\mathrm{T}  \\
        -A & 0 \end{smallmatrix}
        \right)
        \end{math}
        称为Lagrange矩阵。设上述Lagrange矩阵可逆,表示为
        \begin{align*}
        \begin{bmatrix} Q & -A^\mathrm{T}  \\
        -A & 0 \end{bmatrix}^{-1} = \begin{bmatrix} H & -R^\mathrm{T}  \\
        -R & S \end{bmatrix}=I_{m+n}
        \end{align*}
        由此可以
        % \begin{align*}
        % \begin{bmatrix} Q & -A^\mathrm{T}  \\
        % -A & 0 \end{bmatrix}^{-1} = \begin{bmatrix} H & -R^\mathrm{T}  \\
        % -R & S \end{bmatrix}
        % \end{align*}
        推得
        \begin{align*}
         -QH+A^\mathrm{T} R &= I_n \\
         -QR^\mathrm{T} -A^\mathrm{T} S &= O_{m\times n} \\
         -AH &=O_{n\times m}\\
         AR^\mathrm{T} &=I_m
        \end{align*}
        假设$Q^{-1}$存在,则有
        \begin{align*}
        & H = Q^{-1}-Q^{-1}A^\mathrm{T} (Q^{-1}A^\mathrm{T} )AQ^{-1} \\
        & R = (AQ^{-1}A^\mathrm{T} )^{-1}AQ^{-1} \\
        & S = (AQ^{-1}A^\mathrm{T} )^{-1}
        \end{align*}
        在方程组(\ref{最优化算法1})的两边乘上Lagrange矩阵的逆,得到问题的解
        \begin{align*}
        & x^* = -Hc+R^\mathrm{T} b \\
        & {\lambda}^*=Rc-Sb
        \end{align*}
    \subsection{内点算法}
        \par
        Ye.Tse于1989年给出二次规划的内点算法。考虑凸二次规划问题
        \begin{align*}
        & \mathop{\min}\limits_{x} \ f(x) = \frac 12x^\mathrm{T} Qx+c^\mathrm{T} x\\
        & s.t.\left\{
        \begin{aligned}
        &A^\mathrm{T}x=b\\
        &x \geqslant 0
        \end{aligned}
        \right.
        \end{align*}
        设已有$x_k$是一内点,即
        \begin{align*}
             &A^\mathrm{T} x_k=b\\
             &x_k > 0
        \end{align*}
        定义矩阵
        \begin{align*}
        D_k=\begin{bmatrix} (x_k)_1 & \cdots & 0 \\
        \vdots & \ddots & \vdots \\0 & \cdots & (x_k)_n\end{bmatrix} = diag(x_k)
        \end{align*}
        作变量代换$\hat{x}:=T_kx$,如下
        \begin{align*}
        & {\hat{x}}_i=\frac{(n+1)(D_k^{-1}x)_i}{e^\mathrm{T} D_k^{-1}x+1}\quad i=1,2,\ldots,n\\
        & {\hat{x}}_{n+1}=(n+1)/[e^\mathrm{T} D_k^{-1}x+1]
        \end{align*}
        则将原问题转化为
        \begin{align}
        \label{原问题的转化1}
        &\mathop {\min}\limits_{\hat{x}\in R^{n+1}}\  {\hat{x}}_{n+1}Q(T_k^{-1}\hat{x})\\
        &s.t.\left\{
        \begin{aligned}
        &A^\mathrm{T} D_k \hat{x}[n] - \hat{x}_{n+1}b = 0\\
        &e^\mathrm{T}\hat{x} = n+1\\
        &\hat{x}[n] \geqslant 0\\
        &{\hat{x}}_{n+1}>0
        \end{aligned}
        \right.
        \end{align}
        其中:$e=(1,1,\ldots,1)^\mathrm{T} $,$\hat{x}[n]=({\hat{x}}_1,{\hat{x}}_1,\ldots,{\hat{x}}_n)^\mathrm{T} $。
        并且由上面的变量代换,我们有
        \begin{align*}
        x=T_k^{-1}{\hat{x}}=D_k\hat{x}[n]/{\hat{x}}_{n+1}
        \end{align*}
        所以
        \begin{align*}
        &\mathop {\min}\ {\hat{C}}_{k}{\hat{x}}[n]+\frac 12 {\hat{x}}[n]^\mathrm{T} {\hat{Q}}_k{\hat{x}}[n]/{\hat{x}}_{n+1}\\
        &s.t.\left\{
        \begin{aligned}
        &{\hat{A}}_k^\mathrm{T}  {\hat{x}} = \hat{b}\\
        &\hat{x}[n] \geqslant 0\\
        &{\hat{x}}_{n+1}>0
        \end{aligned}
        \right.
        \end{align*}
        其中:
        \begin{align*}
        &{\hat{Q}}_k=D_kQD_k,\quad {\hat{C}}_k=D_kc\\
        &{\hat{A}}_k = \begin{bmatrix}D_k^TA_k&e\\-b^\mathrm{T} &e \end{bmatrix},\quad
        {\hat{b}}=\begin{pmatrix}0\\\vdots\\0\\n+1 \end{pmatrix}
        \end{align*}
        \par
        对于所给的已知迭代点$x_k$,有$\hat{x}=e$。所以我们有理由考虑在$\hat{x}=e$附近求解上述问题。于是,把$\hat{x}[n] \geqslant 0,{\hat{x}}_{n+1}>0$加强为$\|\hat{x}-e\|_2 \leqslant \beta < 1$,$\beta$为与$k$无关的正常数。利用KKT条件,我们知道,求解强化后的问题,等价于求解
        \begin{align}
        &{\hat{C}}_k+{\hat{x}}_{n+1}^{-1}{\hat{Q}}_k{\hat{x}}[n]={\hat{A}}_k[n]\lambda+\mu({\hat{x}}[n]-e)\label{eqa}\\
        &-\frac 12 \frac {1}{{{\hat{x}}_{n+1}}^{2}}{\hat{x}}[n]^\mathrm{T} {\hat{Q}}_k{\hat{x}}[n]=\left( {\hat{a}}_{n+1}^{(k)} \right) ^\mathrm{T} \lambda + \mu({\hat{x}}_{n+1}-1)=0\label{eqb}\\
        &{\hat{A}}_k^\mathrm{T}  {\hat{x}} = {\hat{b}}\label{eqc}\\
        &\|\hat{x}-e\|_2 \leqslant \beta \label{eqd}\\
        &\mu\left[\|\hat{x}-e\|_2 - \beta \right] = 0,\mu \leqslant 0\label{eqe}
        \end{align}
        其中:${\hat{A}}_k[n]$是${\hat{A}}_k$的前$n$行组成的矩阵,${\hat{a}}_{n+1}^{(k)}$是${\hat{A}}_k$的第$n+1$行,将式(\ref{eqa})-(\ref{eqe})写成矩阵形式
        \begin{align}
        \label{矩阵形式}
        P_k\begin{bmatrix}{\hat{x}}[n]\\ \hat{\lambda} \end{bmatrix}={\hat{x}}_{n+1}\hat{b}+\bar{b}
        \end{align}
        其中:
        \begin{align*}
        &P_k = \begin{bmatrix}{\hat{Q}}_k+{\hat{\mu}}I&-{\hat{A}}_k[n]\\{\hat{A}}[n]&0 \end{bmatrix}\\
        &\hat{b} = \begin{bmatrix}{\hat{C}}_k\\b\\-1 \end{bmatrix}\quad \bar{b} = \begin{bmatrix}{\hat{\mu}}_e\\0\\n+1 \end{bmatrix}\\
        &{\hat{\lambda}}={\hat{x}}_{n+1}\lambda,\ \hat{\mu} = -\hat{x}_{n+1}\mu
        \end{align*}
        于是,对任何给定的${\hat{\mu}} \geqslant 0$,我们可由(\ref{矩阵形式})求得${\hat{\lambda}}$和${\hat{x}}[n]$。然后,将求得的${\hat{\lambda}}{\hat{x}}[n]$代入(\ref{eqb})即可得到${\hat{x}}_{n+1}$。于是,对任何${\hat{\mu}} \geqslant 0$,都可求得${\hat{x}}(\hat{\mu})$。定义函数
        \begin{align*}
        h(\hat{\mu})=\|\hat{x}(\hat{\mu})-e\|_2 - \beta
        \end{align*}
        如果$h(0) \leqslant 0$,则$\hat{x}(0)$为上面函数的解。之后,$x=D_k\hat{x}(0)[n]/{\hat{x}(0)}_{n+1}$是原问题的解。如果$h(0)>0$,由于$\mathop {\lim}\limits_{\mu \to \infty}h(\mu)=-\beta<0$,可用方法求解${\hat{\mu}}_k$使得$h({\hat{\mu}}_k)=0$,从而可以得到强化后问题的解${\hat{x}}({\mu}_k)$,将其变换回去即可得到$x_{k+1}$。
        \begin{align*}
        x_{k+1}=T_k^{-1}\hat{x}({\hat{\mu}}_k) = \frac{D_k\hat{x}({\hat{\mu}}_k)[n]}{{\hat{x}({\hat{\mu}}_k)}_{n+1}}
        \end{align*}
        其中:$\hat{x}({\hat{\mu}}_k)[n]=(\hat{x}({\hat{\mu}}_k)_1,\ldots,\hat{x}({\hat{\mu}}_k)_n)^\mathrm{T} $。
\section{MATLAB应用实例}
    \par
    MATLAB中用quadprog命令求解二次规划问题。其调用格式为
    \par
    [x,fval,exitflag,output,lambda]=quadprog(H,f,A,b,Aeq,beq,lb,ub,x0,options)\\
    其中:H为Hesse矩阵;f为c,$c^\mathrm{T} x$;lambda:目标函数在极点$x$处的Lagrange乘子。
    \par
    用quadprog求解如下二次规划问题
    \begin{align*}
    &\mathop {\min}\limits_x \ f=3 x_1^2+2x_2^2+3x_1-4x_2\\
    &s.t.\left\{
    \begin{aligned}
    &2x_1+x_2 \leqslant 4\\
    &-2x_1+2x_2 \leqslant 4\\
    &x_1 \geqslant 0,x_2 \geqslant 0
    \end{aligned}
    \right.
    \end{align*}
    求解程序如下
    \begin{lstlisting}[language=Matlab]
    H = [6,-4;-4,4];
    f = [3,-4];
    A = [2,1;-1,2];
    b = [4;4];
    lb = [0,0];
    ub = [];
    [x, fval] = quadprog(H,f,A,b,[],[],lb,ub)
    \end{lstlisting}
\section{组合投资问题的混合整数规划}
    \par
    前面的引例(\ref{sec:二次规划引例})中,我们介绍了最基本的组合投资模型。下面,我们将进一步引申到混合二次整数规划。对于普通二次规划
    \begin{align*}
    &\mathop {\min}\limits_x \  \lambda x^\mathrm{T} Qx-r^\mathrm{T} x\\
    &s.t.\left\{
    \begin{aligned}
    &\sum\limits_i x_i=1\\
    &0 \leqslant x_i\leqslant 1\\
    &i=1,2,\ldots,n
    \end{aligned}
  \right.
    \end{align*}
    增设变量$v_i$,使
    \begin{align*}
    v_i=\left\{
    \begin{aligned}
    & 0 \quad x_i= 0\\
    & 1 \quad x_i> 0
    \end{aligned}
    \right.
    \end{align*}
    并且要求
    \begin{align*}
    &m \leqslant \mathop {\sum}\limits_i v_i \leqslant M\\
    \end{align*}
    其中:$m,M$为常量。
    \par
    然后,我们要求$x_i$的取值在$f_{min}i$和$f_{max}i$之间。重写组合投资问题为
    \begin{align*}
    &\mathop {\min}\limits_x\ \lambda x^\mathrm{T} Qx-r^\mathrm{T} x\\
    &s.t.\left\{
    \begin{aligned}
    &\mathop {\sum}\limits_i x_i=1\\
    &v_i=0/1\\
    &m \leqslant \mathop {\sum} v_i \leqslant M\\
    &v_if_{min}i \leqslant x_i \leqslant v_if_{max}i
    \end{aligned}
    \right.
    \end{align*}
    \par
    上述问题是以0-1变量$v$的二次规划问题。对于混合规划问题,MATLAB目前为止只能求解混合线性整数规划。所以,我们要想办法将二次规划转化为线性规划。设置一个$z$变量,让$z$来逼近$x^\mathrm{T} Qx$。
    \begin{align*}
    &\mathop {\min}\limits_{x,z} \ \lambda z-r^\mathrm{T} x\\
    &s.t.\left\{
    \begin{aligned}
    &x^\mathrm{T} Qx-z \leqslant 0\\
    &z \geqslant 0\\
    &x^\mathrm{T} e=1\\
    &v=0/1\\
    &m\leqslant v^\mathrm{T} e \leqslant M\\
    &v_if_{min}i \leqslant x_i \leqslant v_if_{max}i
    \end{aligned}
    \right.
    \end{align*}
    \par
    虽然现在已经将目标函数变为线性,但其约束中仍有二次函数。下面,我们来处理$x^\mathrm{T} Qx-z$。将$x^\mathrm{T} Qx-z$在$x_0$处一阶泰勒展开,有
    \begin{align*}
    x^\mathrm{T} Qx-z=x_0^\mathrm{T} Qx_0+2x_0^\mathrm{T} Q\delta-z+O(||\delta||^2)
    \end{align*}
    用$x-x_0$来代替$\delta$,有
    \begin{align*}
    x^\mathrm{T} Qx-z=-x_0^\mathrm{T} Qx_0+2x_0^\mathrm{T} Qx-z+O(|x-x_0|^2)
    \end{align*}
    \par
    现在的$x^\mathrm{T} Qx-z$就变为关于$x$的线性函数。对于$x_k$,有
    \begin{align*}
    -x_k^\mathrm{T} Qx_k+2x_k^\mathrm{T} Qx-z \leqslant 0
    \end{align*}
    将上式写成$Ax \leqslant b$的标准形式,有
    \begin{align*}
    &A=2x_k^\mathrm{T} Q\\
    &b=x_k^\mathrm{T} Qx_k
    \end{align*}
    即,在求$x_{k+1}$时,要利用已知的$x_k$。于是,原混合整数二次规划变为
    \begin{align*}
    &\mathop {\min}\limits_{x,z}\  \lambda z-r^\mathrm{T} x\\
    &s.t.\left\{
    \begin{aligned}
    &Ax-z \leqslant b\\
    &z \geqslant 0\\
    &x^\mathrm{T} e=1\\
    &v=0/1\\
    &m\leqslant v^\mathrm{T} e \leqslant M\\
    &v_i\cdot lb \leqslant x_i \leqslant ub\cdot v_i
    \end{aligned}
    \right.
    \end{align*}
    其中:$A,b$如上定义。MATLAB求解程序如下
    \begin{lstlisting}[language = Matlab]
    %% 混合整数规划求解组合投资问题
    %加载数据
    load port5
    r = mean_return;
    Q = Correlation .* (stdDev_return * stdDev_return');
    N = length(r);
    xvars = 1:N;
    vvars = N+1:2*N;
    zvar = 2*N+1;
    lb = zeros(2*N+1,1);
    ub = ones(2*N+1,1);
    ub(zvar) = Inf;
    M = 150;
    m = 100;
    A = zeros(1,2*N+1); % Allocate A matrix
    A(vvars) = 1; % A*x represents the sum of the v(i)
    A = [A;-A];
    b = zeros(2,1); % Allocate b vector
    b(1) = M;
    b(2) = -m;
    fmin = 0.001;
    fmax = 0.05;
    Atemp = eye(N);
    Amax = horzcat(Atemp,-Atemp*fmax,zeros(N,1));
    A = [A;Amax];
    b = [b;zeros(N,1)];
    Amin = horzcat(-Atemp,Atemp*fmin,zeros(N,1));
    A = [A;Amin];
    b = [b;zeros(N,1)];
    Aeq = zeros(1,2*N+1); % Allocate Aeq matrix
    Aeq(xvars) = 1;
    beq = 1;
    lambda = 100;
    f = [-r;zeros(N,1);lambda];
    options = optimoptions(@intlinprog,'Display','off'); % Suppress iterative display
    [xLinInt,fval,exitFlagInt,output] = intlinprog(f,vvars,A,b,Aeq,beq,lb,ub,options);
    thediff = 1e-4;
    iter = 1; % iteration counter
    assets = xLinInt(xvars); % the x variables
    truequadratic = assets'*Q*assets;
    zslack = xLinInt(zvar); % slack variable value
    while abs((zslack - truequadratic)/truequadratic) > thediff % relative error
        newArow = horzcat(2*assets'*Q,zeros(1,N),-1); % Linearized constraint
        A = [A;newArow];
        b = [b;truequadratic];
        % Solve the problem with the new constraints
        [xLinInt,fval,exitFlagInt,output] = intlinprog(f,vvars,A,b,Aeq,beq,lb,ub,options);
        assets = (assets+xLinInt(xvars))/2; % Midway from the previous to the current
    %     assets = xLinInt(xvars); % Use the previous line or this one
        truequadratic = assets'*Q*assets;
        zslack = xLinInt(zvar);
        history = [history;truequadratic,zslack];
        iter = iter + 1;
    end
    plot(history)
    legend('Quadratic','Slack')
    xlabel('Iteration number')
    title('Quadratic and linear approximation (slack)')
    disp(output.absolutegap)
    bar(xLinInt(xvars))
    grid on
    xlabel('Asset index')
    ylabel('Proportion of investment')
    title('Optimal asset allocation')
    sum(xLinInt(vvars))
    fprintf('The expected return is %g, and the risk-adjusted return is %g.\n',...
        r'*xLinInt(xvars),-fval)
    \end{lstlisting}
% \end{document}
